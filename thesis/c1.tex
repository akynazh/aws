\chapter{绪论}

\section{选题背景与意义}

在全球人口持续增长以及人们对农产品品质与安全关注度日益提升的大背景下,农业的发展面临着前所未有的挑战与机遇。智慧农业作为一种融合了现代信息技术、生物技术、工程技术等多学科技术的新型农业发展模式\cite{赵春江2021智慧农业的发展现状与未来展望},正逐渐成为推动农业现代化进程的关键力量。它通过对农业生产中的各类数据进行实时采集、分析和处理,实现了农业生产的数字化和智能化\cite{李道亮2012物联网与智慧农业},有效提高了农业生产效率,大大降低了生产成本,为解决当前农业发展面临的诸多问题提供了新的思路和方法。

果实称重作为精准农业的重要环节,对于农业生产管理和决策具有不可或缺的作用\cite{罗锡文2016信息技术提升农业机械化水平}。首先,果实称重得到的数据能够为农民提供有关农作物产量的精确信息,帮助他们及时了解农作物的生长状况\cite{翁杨2019基于深度学习的农业植物表型研究综述},进而合理安排农事活动,如施肥、灌溉、病虫害防治等,以提高农作物的产量和质量。其次,通过对果实称重数据进行分析,还可以为农产品的市场定价、销售策略制定以及供应链管理提供有力的数据支持,有助于优化农产品的流通环节,提高农业经济效益\cite{Lipcsei2021AnalysisOA}。此外,在水果采摘环节,精确的称重能够实现水果按重量分级,满足不同消费者的需求,提升水果的市场竞争力\cite{Ji2019}。

然而,传统的果实称重主要依赖人工操作,存在着诸多明显的不足。首先,人工称重需要耗费大量的人力和时间,尤其是在果实收获季节,繁重的称重工作往往使人力成本大幅增加\cite{Jiang2012};其次,由于操作人员的疲劳、疏忽以及读数误差等因素的存在,人工称重获取的数据准确性和可靠性往往难以保证,并因此影响到后续的生产决策\cite{Chen2002};此外,人工称重的方式在数据记录和管理上也较为繁琐,难以实现数据的实时共享和高效分析,无法满足现代精准农业对数据快速处理和深度挖掘的需求\cite{Widagdo2020RecordingSO}。

随着云计算、边缘计算、图像识别等信息技术的飞速发展,将这些技术应用于农业领域,成为推动农业自动化和数字化转型的必然趋势。其一,云计算技术能够实现称重数据的实时采集和传输,打破了时空限制,使得管理者可以随时随地获取和管理数据\cite{李道亮2012物联网与智慧农业}。通过对海量称重数据的分析挖掘,云端软件能够为农业生产提供更加科学、精准的决策支持,如产量预测、市场趋势分析等,有助于农业生产者更好地应对市场变化,降低生产风险\cite{韩佳伟2022装备与信息协同促进现代智慧农业发展研究};其二,传统的云计算在农村地区面临诸多挑战,主要由于网络延迟高、带宽受限等问题,难以满足实时性需求,因此边缘计算成为更为关键和可行的解决方案\cite{aldhaheri2024loracommunicationagriculture40}。边缘计算技术能够实现在边端完成协议转换、信息认证、数据预处理等操作,不仅降低了延迟,也实现了在网络受限情况下称重活动的正常进行\cite{李娜2023基于云边协同计算的粉料仓自称重系统的设计与实现};其三,图像识别技术的应用让果实信息的处理实现自动化,果实的类别只需由云端自动完成识别,无需人工处理,让称重的效率大大提升\cite{Ni2024}。最后,云端软件可与其他农业信息化软件进行集成\cite{刘洋2013基于物联网与云计算服务的农业温室智能化平台研究与应用术},实现农业生产全过程的信息化管理,进一步提升农业生产的智能化水平,推动智慧农业的发展。

基于以上背景,本文研究将深入调研农业生产活动,归纳与分析具体需求,利用云计算、边缘计算、图像识别等软件技术,设计并开发农业果实称重云端软件,实现智能化的农业数据处理、高效的信息管理以及数据的可视化,为智慧农场的果实采摘称重管理提供了一个高效、稳定且功能丰富的解决方案,有助于推动智慧农场的精准收获和数字化管理。 

\section{国内外现状分析}

国外的智慧农业发展起步较早,农业果实称重技术与云端软件的应用研究取得了较为显著的成果。美国作为农业科技强国,在农业领域广泛应用了先进物联网和人工智能技术以提高生产效率\cite{赵春江2021智慧农业的发展现状与未来展望}。例如,美国的一些大型农场采用了高精度的电子称重传感器,能够实时采集果实的重量数据,并通过无线传输技术将数据发送至云端服务器\cite{陈学庚2020农业机械与信息技术融合发展现状与方向}。在加利福尼亚州的部分果园,利用称重设备与云端软件相结合,实现了对水果采摘过程的精准监控和管理\cite{Ampatzidis2011}。美国还研发出基于图像识别和深度学习的果实称重与分级软件,通过对果实图像的分析\cite{Anisha2019FruitRU},不仅能够准确获取果实的重量,还能对果实的品质进行评估,进一步提高了果实称重与分级的智能化水平。此外,美国相关研究利用搭载计算机视觉的智能无人机,实现了对蓝莓灌木和果实的精准识别,从而提升了田间蓝莓数量统计与产量估算的准确性\cite{nguyen2025accuratecropyieldestimation}。

欧洲在农业果实称重云端软件的研究和应用方面同样处于领先地位。德国的一些农业企业开发了集成化的农业生产管理平台,其中包含了果实称重与数据分析模块\cite{Yin2020}。这些平台通过与各种农业设备的连接,实现了果实称重数据的自动采集和实时上传。同时,平台通过大数据分析技术对数据进行深度挖掘,为农业生产提供了全面的决策支持,如预测产量、优化种植方案等\cite{Phate2021}。荷兰则专注于温室农业中的果实称重技术研究,通过在温室中安装传感器网络,实现了对温室作物果实生长过程的全程监测和重量测量\cite{Graaf2004}。为了实现对温室作物的精准种植和高效管理,荷兰一些企业同样利用了云端软件对数据进行分析管理。

随着国内智慧农业的快速发展,农业果实称重云端软件的研究和应用也逐渐受到重视。近年来,国内的高校和科技企业在该领域积极开展研究,取得了诸多成果,推动了相关技术的持续发展与应用落地。一些高校的研究团队开发了基于物联网的果实称重软件,通过将电子秤与物联网模块相结合,实现了果实称重数据的无线传输和远程监控\cite{Zhu2013}。并且山东的部分果园已经率先应用了这种物联网称重系统。通过该系统,果农可以在手机 APP 实时查看果实的称重数据,方便了果实采摘和销售管理\cite{Gao2023}。国内企业也积极投入到农业果实称重云端软件的研发和推广中。一些科技公司推出了针对不同农业场景的果实称重解决方案\cite{Ningbo2019},不仅提供了硬件设备,还开发了相应的软件平台,实现了数据的云端存储、分析和可视化展示,帮助农业生产者更好地管理果实称重数据,提高农业生产效率。

然而,目前的果实称重软件仍存在一些不足之处。一方面,部分果实称重软件的精度和稳定性有待提高,在复杂的农业生产环境中,容易受到温度、湿度、电磁干扰等因素的影响,导致称重数据出现偏差\cite{汤建华2018}。另一方面,现有的云端软件在数据安全和隐私保护方面还存在一定的风险,随着农业数据的价值不断提升,如何保障数据的安全传输和存储,防止数据泄露和篡改,成为亟待解决的问题。此外,不同地区和不同作物的果实称重需求存在差异,现有的软件在通用性和适应性方面还需要进一步优化,以满足多样化的农业生产需求,比如云端软件需要支持更多物联网通信协议,如 MQTT、HTTP、CoAP 等,以满足更多不同类型电子秤的通信。

\section{论文工作内容}

本文研究深入调研农业生产活动,应用软件工程相关知识理论,归纳与分析具体需求,利用先进软件技术,设计出合理的系统架构并可开发农业果实称重云端软件,实现高性能且稳定可用的智能化农业数据处理流程、高效且丰富的信息管理功能以及多个维度下数据的可视化。论文的工作如下:

(1)背景调研和技术概述。对农业果实称重云端软件的研究背景进行阐述,分析其在农业生产中的重要性和应用前景,并对国内外相关研究现状进行综述。同时,对软件设计和开发过程中涉及到的理论和相关技术进行概述。

(2)软件需求与可行性分析。调研农业生产活动,进行需求分析与可行性分析,其中需求分析包括功能需求分析和非功能需求分析,可行性分析包括技术可行性、经济可行性和社会可行性三个方面的分析。

(3)软件设计与实现。在需求分析的基础上,选用当下可靠的框架和技术,进行软件架构设计、数据库设计和各个功能模块的设计与实现。

(4)软件部署与测试。部署软件并完成功能测试和性能测试,对测试结果进行理论分析,给出相关结论。

\section{论文组织结构}

论文总共分为六章,各章主要内容如下:

第一章:绪论。本章节主要阐述农业果实称重云端软件的研究背景及选题的意义,重点介绍农业果实称重云端软件的发展及项目开发的意义,最后阐述论文主要的工作内容和组织结构。

第二章:软件相关技术概述。列举并简要描述了软件设计和开发过程中涉及到的理论和相关技术,包括 Spring 技术栈、YOLO 目标检测算法、EMQX 网关框架等,对各个技术的原理和优势特点进行论述。

第三章:软件需求与可行性分析。通过对软件需求的深入分析,明确了包括称重、用户、果实、作业四个模块的功能需求,明确了软件的性能、安全、部署方式等非功能需求,从技术、经济和社会三个角度评估了软件的可行性。

第四章:软件设计与实现。阐述软件架构设计、数据库设计以及各个功能模块的设计与实现,描述和分析软件的设计与实现思路以及如何通过合理的架构和设计来支持系统的高效运行。

第五章:软件测试与分析。给出软件的实现成果,给出各项功能的测试过程和测试结果,针对测试结果进行分析并给出结论。

第六章:总结与展望。总结了本文的主要工作,指出农业果实称重云端软件中存在的不足及可行的改进方向。
