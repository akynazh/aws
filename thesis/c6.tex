\chapter{总结与展望}

\section{论文工作总结}

本文研究深入调研农业生产活动,应用软件工程相关知识理论,归纳与分析具体需求,利用先进软件技术,设计出合理的系统架构并可开发农业果实称重云端软件,实现高性能且稳定可用的智能化农业数据处理流程、高效且丰富的信息管理功能以及多个维度下数据的可视化。论文的工作如下:

(1)背景调研和技术概述。对农业果实称重云端软件的研究背景进行阐述,分析其在农业生产中的重要性和应用前景,并对国内外相关研究现状进行综述。同时,对软件设计和开发过程中涉及到的理论和相关技术进行概述,包括 Spring 技术栈、YOLO 目标检测算法、EMQX 网关框架等。

(2)软件需求与可行性分析。调研农业生产活动,主要分析了果实收获和称重入库的具体流程,进行需求分析与可行性分析,其中需求分析包括功能需求分析和非功能需求分析,可行性分析包括技术可行性、经济可行性和社会可行性三个方面的分析。

(3)软件设计与实现。在需求分析的基础上,选用当下可靠的框架和技术,进行了软件架构设计、数据库设计、称重模块的功能设计与实现、果实模块的功能设计与实现、作业模块的功能设计与实现和用户模块的功能设计与实现。

(4)软件部署与测试。通过使用容器技术组织各个应用,在本地完成部署,然后完成功能测试和性能测试,对测试结果进行理论分析,给出相关结论。

\section{后续工作展望}

通过软件的设计、实现和测试,本文的农业果实称重云端软件可以满足中小型农场的使用,能够保证数据的准确性和可靠性,为农业生产提供有力的支持。但是,本文的研究仍然存在一些不足之处,主要体现在以下几个方面:

(1)软件所实现的 EMQX 静态集群在使用上存在一定的局限性。静态集群在伸缩性和灵活性上较差,无法更高效地增删 EMQX 节点,以满足扩展变更需求。后续工作中,可以考虑基于 Mnesia 之类的分布式数据库,实现一个 EMQX 动态集群,为农业果实称重云端软件提供更加强大的数据处理能力。

(2)在数据存储方面存在一定的局限性。随着业务不断扩展,系统产生的数据会以数量级增长,存储方面,本软件采用了 MySQL,但随着数据量的增加,单机数据库的性能瓶颈逐渐显现。为应对这一挑战,后续工作中可以考虑进行数据库分片,通过水平分片将数据分散到多个数据库实例上,以提高查询性能和扩展性。此外也可以考虑进行数据归档与清理,定期归档历史数据,清理过期或不常用的数据,减轻数据库负担。通过这些优化措施,能够有效支持系统的可扩展性和高性能需求。

(3)深入挖掘称重数据的应用价值。未来可结合采集的果实图像训练质量分级模型,实现果实品质评估与智能化分级,提升数据利用效率与管理智能化水平。