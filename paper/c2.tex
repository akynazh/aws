\chapter{软件需求分析}

\section{功能需求分析}

本研究从实际农业生产流程出发,了解果实称重业务的实际流程,分析现有业务存在的问题和痛点,明确软件的功能需求,并完成功能模块划分和工作分解。

在农场中,传统的果实称重模式主要依赖人工操作。采摘员工将果实放置在称重台上,称重人员完成读数并记录称重结果。在这种传统的称重模式下,存在的很多问题和痛点:

1、人工称重的方式数据不可靠。称重人员在称重读数时存在疲劳、疏忽以及读数误差等人为因素的存在,获取的数据准确性难以保证。不可靠的数据将会影响到后续的生产决策;

2、人工称重的方式在数据记录和管理上也较为繁琐。人工记录数据的方式无法实现数据的实时共享和高效分析,无法满足现代精准农业对数据快速处理和深度挖掘的需求;

3、人工称重需要耗费大量的人力和时间,生产效率低下。

面对这些痛点和问题,可以针对性地进行解决,设计一个更高效、更智能的称重流程,如称重流程时序图\ref{fig:称重流程时序图}所示。

\begin{figure}[H]
    \centering
    \includegraphics[width=0.8\linewidth]{../design/out/称重流程时序图.png}
    \caption{称重流程时序图}
    \label{fig:称重流程时序图}
\end{figure}

图\ref{fig:称重流程时序图}中展现了三个称重流程中所涉及到的对象,分别是采摘人员、电子秤和云端软件。称重流程具体是:

1、采摘人员放置果实至电子秤处,电子秤开始读取称重数据;

2、采摘人员扫描员工卡,电子秤读取卡
数据;

3、电子秤拍摄果实照片;

4、电子秤生成并提交称重信息至云端软件;

5、云端软件处理称重信息并返回结果;

6、电子秤显示提交结果,果实完成入库。

上述操作自动化了称重信息的处理和存储,采摘人员只需刷卡即可完成所有称重步骤,后台收到数据后可以实时地完成数据的存储和分析。

基于这个称重流程,可以分析并归纳出五个模块的功能需求,分别是称重模块、用户模块、果实模块、作业模块。如工作分解结构图\ref{fig:工作分解结构图}所示。

\begin{figure}[H]
    \centering
    \includegraphics[width=0.8\linewidth]{../design/out/工作分解结构图.png}
    \caption{工作分解结构图}
    \label{fig:工作分解结构图}
\end{figure}

(1)称重模块:该模块提供电子秤查询和管理、称重记录处理和查询、待办记录查询和处理以及称重记录统计分析功能。对于称重记录处理功能,电子秤终端可以发送称重数据到云端软件,完成称重记录的存储;对于待办记录处理功能,后台处理失败的称重数据,将其转存至待办记录处,在后台管理处由人工进行处理;对于称重记录统计分析功能,可以得出多个维度的数据统计结果,至少包括果实年产量、各采摘作业批次果实产量、用户年产出、各采摘作业批次用户产出等统计数据,这些统计数据需要支持导出为 Excel 并可以在前台通过折线图进行展示。

(2)用户模块:该模块提供用户认证、用户授权以及用户管理功能;普通员工在这个模块中,可以进行登录认证到后台个人中心、更新个人信息、查看个人产出等功能;管理员是一种特殊的用户,在这个模块中,管理员除了普通用户所拥有的功能以外,还可以对用户、果实、电子秤、作业、待办记录等对象进行管理;电子秤也作为一种特殊的用户存在,电子秤提交称重数据,也需要经过该模块的密码认证以确保数据安全可靠。

(3)果实模块:包含果实查询、管理和图像识别功能。对于果实图像识别功能,云端软件在后台可以根据果实图像识别出果实种类,如果可信度超过一定的阈值则判定为识别成功,完成称重数据处理并持久化,否则将称重数据转存至待办记录处。

(4)作业模块:包含采摘作业查询和管理功能。各种果实拥有各自的成熟期,也就有着不同的采摘批次。因此定义采摘作业为某个时间段对某个果实的采摘作业。通过该模块,管理员可以进行对采摘作业的查询和维护。

根据上述功能模块的分析,可以进一步得出软件中存在的角色,分别是员工、管理员和电子秤,需要在后续的开发中做好权限控制,如表\ref{tab:uc1_permissions}和表\ref{tab:uco_permissions}所示。

\begin{table}[ht]
\centering
\begin{tabular}{|c|c|c|c|c|c|c|c|}
\hline
角色 & 电子秤查询/管理 & 记录处理 & 记录查询 & 统计分析 & 待办查询/处理 \\
\hline
采摘员工 & 否 & 否 & 否 & 是 & 否 \\
\hline
电子秤 & 否 & 是 & 否 & 否 & 否 \\
\hline
农场管理员 & 是 & 否 & 是 & 是 & 是 \\
\hline
\end{tabular}
\vspace{10pt}
\caption{称重模块角色权限表}
\label{tab:uc1_permissions}
\end{table}

\begin{table}[ht]
\centering
\begin{tabular}{|c|c|c|c|c|c|c|}
\hline
角色 & 用户管理 & 果实查询 & 果实管理 & 果实图像识别 & 作业查询 & 作业管理 \\
\hline
采摘员工 & 否 & 是 & 否 & 否 & 是 & 否 \\
\hline
电子秤 & 否 & 否 & 否 & 是 & 否 & 否 \\
\hline
农场管理员 & 是 & 是 & 是 & 否 & 是 & 是 \\
\hline
\end{tabular}
\vspace{10pt}
\caption{其余模块角色权限表}
\label{tab:uco_permissions}
\end{table}

表\ref{tab:uc1_permissions}和表\ref{tab:uco_permissions}中,左侧第一列记录角色名称,其余列记录功能及是否拥有对应权限。

\section{非功能需求分析}

注意需要实现多种协议电子秤的兼容,包括 HTTP、CoAP、MQTT 等

非系统功能需求是指系统在使用过程中,除了实现基本功能外,还需要满足的一些性能、安全、可靠性、易用性等方面的需求。本系统的非系统功能需求主要包括以下几个方面:

1、高可用性:称重数据的提交和查询需要保证系统的高可用性,确保系统能够稳定运行,不会因为系统故障或网络中断导致数据丢失。
2、高性能:系统需要具备良好的性能,能够在高并发、大数据量的情况下保持稳定运行,具备良好的响应速度和吞吐量。
3、数据安全:系统需要保证称重数据的安全性,确保数据传输过程中的加密和解密,防止数据泄露和篡改。
4、易用性:系统需要具备良好的用户界面,操作简便,易于推广使用,用户可以快速上手,提高工作效率。
5、可扩展性:系统需要具备良好的可扩展性,能够支持多种电子秤通信协议,满足不同类型电子秤的通信需求。

其中,尤其需要注意数据安全、高性能和高可用性。下面对这三个需求进行具体分析。

对于数据安全,首先需要确保所有称重数据的提交都是加密的,比如使用 HTTPS 代替 HTTP,来完成数据的通信;其次,对于数据库的访问,需要配置好用户名和密钥,禁止匿名访问;第三,存储在数据库的密钥需要加密,不可以明文存储;最后,对于一些数据处理中间件,如果涉及某些主题的订阅和消息发布,不同的消费者和生产者需要做好权限的控制,避免消息的错误发布和消费。

对于高性能,在农场称重的活动中,同一时间,可能有上万台设备在提交称重数据,这时候的 QPS 将是上万级别的,如果不能处理好称重数据的快速提交,将会对农业生产造成很大的影响。因此,需要确保系统的高可用性,可以采用一些消息队列来处理这个问题。

对于高可用性,如果在农场称重过程中,后台服务出现宕机的情况,那么将无法处理任何请求,这将对农业生产造成很大的影响。因此,需要确立好一个良好可伸缩的软件架构,在某台机器宕机的情况下,可以快速使用另一个机器进行服务,确保称重服务的高可用。

\section{可行性分析}

可行性分析是一门研究技术领域中经济问题和经济规律的科学,旨在探索在特定技术条件下如何提高经济效益。它是技术与经济的交叉领域,主要研究技术实施的经济效果以及技术与经济的最佳组合方案。首先,需要对系统进行全面的可行性分析,以确定系统是否能够实现,是否满足各方面的需求。可行性分析包括技术可行性、经济可行性、法律和社会可行性,还需考虑增加操作可行性的实际需求。

1、操作可行性分析

农业果实称重云端软件的操作界面设计力求简洁直观,简化操作流程,使其尽可能符合用户的使用习惯和业务需求。因此,操作人员只需经过简单的培训即可熟练使用,从而确保本系统在操作可行性方面不会存在任何问题。

2、技术可行性分析

农业果实称重云端软件属于信息管理系统,主要涉及数据库开发技术和前端操作界面的开发。当前,数据库开发技术已取得显著进步,结合本系统的应用范围和功能需求,采用合适的开发平台可以充分满足功能要求。此外,数据库开发与前端界面结合在信息管理系统领域已有众多成功案例,因此本系统在技术实现上是完全可行的。

系统的可行性分析主要从以下几个方面进行考虑:
	
1、安全性:
    为了确保系统的安全性,利用数据库技术,通过编写存储过程等手段,保障系统的并发性和恢复能力,同时有效防止通过SQL注入等技术对数据进行窃取和篡改。
2、可运行性:
    本系统作为一个小型管理系统,所需的资源消耗非常小,能够在较低配置的环境下稳定运行。

3、经济可行性分析

由于本软件是一个小型的农业果实称重云端软件,因此其开发经费和投资较少,仅需一台普通的计算机即可完成开发。系统的开发过程并不复杂,操作尽可能保持简洁易用,具有良好的用户体验和实用性,因而在经济上具有合理性。

在系统开发方面,由一人独立完成,预计开发周期约为三个月。开发成本主要包括人工费用,此外还包括计算机和软件的投入成本。由于计算机和软件均可重复使用,采用自主开发方式不仅节省了费用,还能根据自身需求定制功能,并掌握完整的源代码,方便系统后续的扩展和维护。因此,从成本和效益的角度来看,开发农业果实称重云端软件是完全可行的。

\section{本章小结}

本章主要对系统的需求进行了详细分析,包括系统的功能需求和非功能需求,以及系统的可行性分析。通过对系统功能的深入分析,明确了各模块的作用与需求,包括用户管理、果实管理、工作服务模块和称重服务模块。我们通过分析称重流程及其关键环节,确定了系统需要实现的功能和技术要求,同时明确了系统的安全性、性能、可用性和扩展性等非功能性需求。

在可行性分析方面,从操作、技术和经济三个角度评估了本系统的可行性。系统在操作层面简洁易用,符合农业操作员的工作习惯;在技术层面,结合现代数据库技术和前端开发经验,系统具备实现的技术基础;在经济层面,考虑到开发成本和效益,本系统的开发具有较高的性价比。

综上所述,本章为系统设计提供了全面的理论依据,确定了系统的核心需求与可行性,为后续的系统设计与开发奠定了坚实的基础。
