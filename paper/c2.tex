\chapter{软件相关技术概述}

\section{MVC/MVVM 设计模式}

在农业果实称重云端软件中,基于前后端分离的开发思想,采用了 Spring + Vue 的技术栈,基于 MVC(Model-View-Controller) 和 MVVM(Model-View-ViewModel) 的设计模式来完成开发。下面对这两种设计模式进行具体阐述。

在现代 Web 开发中,前后端分离已成为一种主流的架构模式。传统的 Web 开发通常采用服务端渲染(SSR)的方式,前端和后端集成在一起,由后端生成 HTML 页面并返回给浏览器,浏览器仅负责解析和展示页面内容。然而,随着 Web 技术的发展和前端框架(如 React、Vue.js、Angular)的广泛应用,前后端分离的架构逐渐成为主流。

在前后端分离架构下,前端和后端的开发、部署和运行环境是独立的,二者通过 API 进行通信。前端专注于页面的展示和交互,通过调用后端的接口获取数据和提交操作;后端则专注于业务逻辑的处理、数据存储和安全管理。通过这种分离,前后端可以独立开发、测试和部署,提高了开发效率和系统的扩展性。

在前后端分离的架构模式下,MVC 设计模式依然存在,但它的角色和应用场景发生了一些变化。前后端分离的架构通常意味着前端和后端的职责和运行环境是分开的,前端通过 API(通常是 RESTful 或 GraphQL)与后端进行通信。

后端依然可以使用 MVC 模式来处理请求和响应。具体来说,后端的 MVC 主要处理以下内容:
1、Model:表示业务逻辑层或数据模型,通常是数据库实体或服务层中的数据对象。它负责处理数据存取、验证和计算等操作。
2、View:在传统的 MVC 中,View 是用户界面部分,负责显示数据。而在前后端分离架构中,后端的 View 主要是指服务器端生成的 JSON 或 XML 数据,它不是直接渲染页面,而是以 API 的形式将数据返回给前端。
3、Controller:处理客户端请求、调用相应的业务逻辑(Model),并最终返回一个响应。Controller 将从 Model 中获取数据,通常会将其转化为 JSON 格式并返回给前端。

在这种情况下,后端的 MVC 模式 用于接收和处理 API 请求,完成业务逻辑并返回数据,而不直接负责页面渲染。后端的 Controller 负责接收前端发送的 HTTP 请求,调用 Model 层进行数据处理,然后通过 API 将数据返回给前端。

前后端分离架构中的前端通常使用 MVVM(Model-View-ViewModel)或类似的设计模式,而不完全遵循传统的 MVC 模式。前端通常不负责数据的存储与计算,而是通过 API 从后端获取数据并展示。

1、Model:在前端中,Model 主要指代数据模型,通常通过 API 获取的数据。例如,前端通过 HTTP 请求从后端获取的 JSON 数据。
2、View:表示用户界面(UI),前端的页面、组件或视图,由前端框架(如 React、Vue、Angular)渲染。
3、ViewModel:前端的 ViewModel 是将 Model 和 View 连接起来的部分,负责将数据处理成适合显示的格式,并监听用户的交互操作。前端框架(如 React、Vue)通常在这个层次进行数据绑定与更新。

在前后端分离的架构中,前端不直接处理数据存储或计算,而是依赖后端提供 API 接口来获取数据。前端通过调用后端 API,获取数据并根据这些数据更新 View(页面)。

对于本项目,在 Spring 后端项目中,采用 MVC 模式来设计 RESTful API。后端 Controller 接收前端的 HTTP 请求,调用 Service 层(Model)处理数据,返回一个包含数据的 JSON 响应。在 Vue 前端项目中,使用 MVVM 模式,Model 是从后端 API 获取的 JSON 数据,View 是用户界面,ViewModel 是控制数据展示和用户交互的逻辑代码。
