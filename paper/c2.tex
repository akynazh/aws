\chapter{系统分析}

\section{系统需求分析}

\subsection{系统功能需求分析}

本研究从实际农业生产流程出发,了解在果实称重管理方面的实际需求和业务流程,分析现有果实称重方式存在的问题和痛点,明确系统的功能需求、性能需求、安全需求以及用户体验需求等。

随着电子秤的推广,在智慧农场果蔬采摘称重是精准收获中重要的一环。而目前很大程度上仍然需要通过人工进行采摘,因此,需要对采摘果实进行称重记录、统计果蔬产量进行管理。传统的果实称重方式主要依赖人工操作,存在着诸多明显的不足。比如无法保证称重数据的准确性和可靠性,人工称重效率低下,数据记录和管理上也较为繁琐,难以实现数据的实时共享和高效分析,比如在一个大型农场信息管理系统中,通常需要对果实称重数据进行统计分析,生成报表,以便于管理者更好地了解农场的生产情况,从而更好地优化生产策略,提高经济效益。

针对以上需求背景,本研究从分析称重流程出发,逐步介绍系统所需要实现的功能点。下面开始模拟并分析称重的流程。

称重人员只需要将果实放上称重台,在终端通过刷卡认证并选择果实种类,然后等待称重数据上传。电子秤终端会自动读取电子秤的称重结果,然后生成更具体的数据信息,最后通过某种通信协议请求云端软件的服务地址,完成称重数据的存储。后台数据处理完成后,给前台反馈处理的结果。具体的称重流程时序图如图\ref{fig:称重流程时序图}所示。

\begin{figure}[H]
    \centering
    \includegraphics[width=0.8\linewidth]{../design/out/称重流程时序图.png}
    \caption{称重流程时序图}
    \label{fig:称重流程时序图}
\end{figure}

对于整个称重流程,有几个关键的判断节点需要注意。首先是员工认证,如果后台认证失败,员工不可进入果实选择界面;接着是称重数据的提交,在后台完成相关动作之后,需要返回给前台相应的提示信息;最后是提示员工是否继续称重,如果不继续称重,需要登出帐号,以确保数据安全,否则退回到果实选择界面。具体的称重流程活动图如图\ref{fig:称重流程活动图}所示。

\begin{figure}[H]
    \centering
    \includegraphics[width=0.8\linewidth]{../design/out/称重流程活动图.png}
    \caption{称重流程活动图}
    \label{fig:称重流程活动图}
\end{figure}

基于此称重流程,可以发掘出几个功能点,首先是用户管理,需要有一个用户管理模块,来完成称重人员的信息认证,除此之外,需要实现管理员和用户的角色区分,管理员可以执行对其他用户的管理功能,以及对诸如电子秤、采摘作业等的管理;其次是果实管理,发送的称重数据需要包括果实的种类,同时后续也需要根据果实的称重结果进行产量相关的统计;接着是工作管理,每次提交的称重数据,应该包含唯一确定一次采摘作业的编号,这样才可以达到分批次统计产量的目的;最后是称重相关的功能,包括电子秤管理和称重数据的处理,注意需要实现多种协议电子秤的兼容,包括 HTTP、CoAP、MQTT 等。

针对上述对称重流程的分析,本系统将会实现一个更加全面的农场信息管理系统,包括果实称重数据的实时采集、传输、存储和处理,果实、员工、采摘作业等信息的管理和可视化展示,帮助农业生产者更好地管理农业生产环境,提高农业生产效率。具体包括了用户管理模块、果实模块、工作服务模块、称重服务模块等。具体模块功能如工作分解结构图\ref{fig:工作分解结构图}所示。

\begin{figure}[H]
    \centering
    \includegraphics[width=0.8\linewidth]{../design/out/工作分解结构图.png}
    \caption{工作分解结构图}
    \label{fig:工作分解结构图}
\end{figure}

工作分解结构图\ref{fig:工作分解结构图}归纳了各个模块应该实现的功能。下面进行具体阐述:

一、用户管理模块,包含用户功能和管理员功能,用户在这个模块中,可以进行登录认证,以及对个人信息的修改。管理员是一种特殊的用户,除了普通用户所拥有的功能以外,管理员可以对其他用户进行增删改查操作。

二、果实模块,包含果实管理功能和果实统计功能。对于果实管理,管理员可以对果实信息进行维护,所有用户可以对果实进行查询。对于果实统计,管理员可以获取任一果实的年产量统计结果和分批作业的产量统计结果,这些统计结果需要支持导出为 Excel 并可以在前台通过折线图完成可视化功能。

三、工作服务模块,包含工作管理功能和工作分配功能。对于工作管理,管理员可以进行对采摘工作的查询和维护,这个采摘工作可以定义为某个时间段对某个果实的采摘作业。对于工作分配,管理员可以分配具体的采摘作业给某些员工,而员工可以查询到自己所负责的作业。

四、称重服务模块,包含电子秤管理功能、称重记录功能和统计分析功能。对于电子秤管理,管理员可以进行电子秤信息的维护和查询。对于称重记录,电子秤终端可以发送称重数据到云端软件,完成称重记录的存储。管理员可以针对任一电子秤,查询其历史称重数据。对于统计分析,员工可以查询自己各作业的采摘量。

分解完系统所涉及的功能模块后,再具体对系统中存在的角色以及它们与系统功能的关联进行阐述。下面通过用例图来对其进行解释,如图\ref{fig:核心功能用例图}所示。

\begin{figure}[H]
    \centering
    \includegraphics[width=0.8\linewidth]{../design/out/核心功能用例图.png}
    \caption{核心功能用例图}
    \label{fig:核心功能用例图}
\end{figure}

从用例图\ref{fig:核心功能用例图}可以知道,本系统存在三个角色:采摘员工、电子秤终端和农场管理员。采摘员工涉及三个方面的功能,包括用户服务、称重服务和数据可视化。电子秤终端则是一个特殊的角色,用来完成称重服务;农场管理员也涉及三个方面的功能,包括用户服务、数据可视化和农场管理。

\subsection{非系统功能需求分析}

非系统功能需求是指系统在使用过程中,除了实现基本功能外,还需要满足的一些性能、安全、可靠性、易用性等方面的需求。本系统的非系统功能需求主要包括以下几个方面:

\begin{enumerate}
    \item 高可用性:称重数据的提交和查询需要保证系统的高可用性,确保系统能够稳定运行,不会因为系统故障或网络中断导致数据丢失。
    \item 高性能:系统需要具备良好的性能,能够在高并发、大数据量的情况下保持稳定运行,具备良好的响应速度和吞吐量。
    \item 数据安全:系统需要保证称重数据的安全性,确保数据传输过程中的加密和解密,防止数据泄露和篡改。
    \item 易用性:系统需要具备良好的用户界面,操作简便,易于推广使用,用户可以快速上手,提高工作效率。
    \item 可扩展性:系统需要具备良好的可扩展性,能够支持多种电子秤通信协议,满足不同类型电子秤的通信需求。
\end{enumerate}

其中,尤其需要注意数据安全、高性能和高可用性。下面对这三个需求进行具体分析。

对于数据安全,首先需要确保所有称重数据的提交都是加密的,比如使用 HTTPS 代替 HTTP,来完成数据的通信;其次,对于数据库的访问,需要配置好用户名和密钥,禁止匿名访问;第三,存储在数据库的密钥需要加密,不可以明文存储;最后,对于一些数据处理中间件,如果涉及某些主题的订阅和消息发布,不同的消费者和生产者需要做好权限的控制,避免消息的错误发布和消费。

对于高性能,在农场称重的活动中,同一时间,可能有上万台设备在提交称重数据,这时候的 QPS 将是上万级别的,如果不能处理好称重数据的快速提交,将会对农业生产造成很大的影响。因此,需要确保系统的高可用性,可以采用一些消息队列来处理这个问题。

对于高可用性,如果在农场称重过程中,后台服务出现宕机的情况,那么将无法处理任何请求,这将对农业生产造成很大的影响。因此,需要确立好一个良好可伸缩的软件架构,在某台机器宕机的情况下,可以快速使用另一个机器进行服务,确保称重服务的高可用。

\section{系统可行性分析}

可行性分析是一门研究技术领域中经济问题和经济规律的科学,旨在探索在特定技术条件下如何提高经济效益。它是技术与经济的交叉领域,主要研究技术实施的经济效果以及技术与经济的最佳组合方案。首先,需要对系统进行全面的可行性分析,以确定系统是否能够实现,是否满足各方面的需求。可行性分析包括技术可行性、经济可行性、法律和社会可行性,还需考虑增加操作可行性的实际需求。

\subsection{操作可行性分析}

农业果实称重云端软件的操作界面设计力求简洁直观,简化操作流程,使其尽可能符合用户的使用习惯和业务需求。因此,操作人员只需经过简单的培训即可熟练使用,从而确保本系统在操作可行性方面不会存在任何问题。

\subsection{技术可行性分析}

农业果实称重云端软件属于信息管理系统,主要涉及数据库开发技术和前端操作界面的开发。当前,数据库开发技术已取得显著进步,结合本系统的应用范围和功能需求,采用合适的开发平台可以充分满足功能要求。此外,数据库开发与前端界面结合在信息管理系统领域已有众多成功案例,因此本系统在技术实现上是完全可行的。

系统的可行性分析主要从以下几个方面进行考虑:
	
\begin{enumerate}
    \item 安全性:
    为了确保系统的安全性,利用数据库技术,通过编写存储过程等手段,保障系统的并发性和恢复能力,同时有效防止通过SQL注入等技术对数据进行窃取和篡改。
    \item 可运行性:
    本系统作为一个小型管理系统,所需的资源消耗非常小,能够在较低配置的环境下稳定运行。
\end{enumerate}

\subsection{经济可行性分析}

由于本软件是一个小型的农业果实称重云端软件,因此其开发经费和投资较少,仅需一台普通的计算机即可完成开发。系统的开发过程并不复杂,操作尽可能保持简洁易用,具有良好的用户体验和实用性,因而在经济上具有合理性。

在系统开发方面,由一人独立完成,预计开发周期约为三个月。开发成本主要包括人工费用,此外还包括计算机和软件的投入成本。由于计算机和软件均可重复使用,采用自主开发方式不仅节省了费用,还能根据自身需求定制功能,并掌握完整的源代码,方便系统后续的扩展和维护。因此,从成本和效益的角度来看,开发农业果实称重云端软件是完全可行的。

\section{本章小结}

本章主要对系统的需求进行了详细分析,包括系统的功能需求和非功能需求,以及系统的可行性分析。通过对系统功能的深入分析,明确了各模块的作用与需求,包括用户管理、果实管理、工作服务模块和称重服务模块。我们通过分析称重流程及其关键环节,确定了系统需要实现的功能和技术要求,同时明确了系统的安全性、性能、可用性和扩展性等非功能性需求。

在可行性分析方面,从操作、技术和经济三个角度评估了本系统的可行性。系统在操作层面简洁易用,符合农业操作员的工作习惯;在技术层面,结合现代数据库技术和前端开发经验,系统具备实现的技术基础;在经济层面,考虑到开发成本和效益,本系统的开发具有较高的性价比。

综上所述,本章为系统设计提供了全面的理论依据,确定了系统的核心需求与可行性,为后续的系统设计与开发奠定了坚实的基础。

%%%%%%%%%%%%%%%%%%%%%%%%%%%%%%%%
