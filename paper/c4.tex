\chapter{软件设计与分析}

\section{软件架构设计}\label{sec:architecture}

农业果实称重云端软件的整体架构采用边端+远端协同的模式进行设计,将靠近农场端的服务称为边端服务,远离农场端的服务称为远端服务,其中边端服务一般部署在农场局域网内,在网络受限的情况下,确保称重流程的正常进行。整体软件部署架构图如图\ref{fig:软件部署架构图}所示。

\begin{figure}[H]
    \centering
    \includegraphics[width=0.8\linewidth]{../design/out/软件部署架构图.png}
    \caption{软件部署架构图}
    \label{fig:软件部署架构图}
\end{figure}

如图\ref{fig:软件部署架构图}所示,存在三个端,分别是边端、远端以及前台网页端,它们分别部署在三台不同服务器上。电子秤提交称重数据的通信流程如下:

1、不同类型的电子秤通过不同协议向网关提交称重数据;

2、网关处理电子秤的发出的请求,通过 EMQX 协议转换技术完成协议转换,通过读取 MySQL 从库数据完成电子秤信息认证和授权,最后将数据以 MQTT 协议的格式发送至 MQTT Broker 集群;

3、MQTT Broker 接收到 MQTT 消息后,该集群内任一节点都可以处理请求并将消息持久化;

4、远端服务订阅 MQTT Broker 集群中所有节点的消息,收到消息后进行消息解析,通过果实识别服务识别出果实种类,然后将称重结果写入 MySQL 主库;

5、至此已经完成称重数据的处理。此外,在前台网页端,员工可以查看历史称重记录,管理员可以在待办界面处理待办,手动填写并提交处理失败的称重数据。

进一步分析软件架构的特点,可以得到软件的分层架构图,如图\ref{fig:软件分层架构图}所示。

\begin{figure}[H]
    \centering
    \includegraphics[width=0.8\linewidth]{../design/out/软件分层架构图.png}
    \caption{软件分层架构图}
    \label{fig:软件分层架构图}
\end{figure}

软件可以抽象出五个层级,分别是表现层、通信层、网关层、服务层、数据层。各层作用如下:

1、表现层:该层可以理解为接口调用者,包含前台网页端和电子秤,两者都通过调用后台提供的接口来完成数据的处理或展示;

2、通信层:该层服务于表现层和网关层之间的通信。软件所支持的通信协议有四种,分别是 HTTP/HTTPS、MQTT、CoAP 和 STOMP。其中前台网页端可以通过 HTTP/HTTPS 调用后台接口,而电子秤可以通过四种协议中任一一种来提交称重数据;

3、网关层:该层提供了接口路由和协议转换功能。表现层发来的请求根据接口路由导向服务层中对应的服务,协议转换功能则服务于电子秤,将不同协议统一转为 MQTT 协议,以 MQTT 协议继续进行后续通信;

4、服务层:该层完成具体的数据服务逻辑。包含四大模块,用户模块、果实模块、作业模块以及称重模块,各个模块相互依赖,共同协作,完成数据的查询和处理;

5、数据层:该层完成数据的持久化和缓存操作。MySQL 数据库用于持久化数据,Redis 数据库用于实现数据缓存功能,而 MQTT Broker 类似于消息队列,用于临时存储 MQTT 消息。

综上所述,根据软件部署架构图\ref{fig:软件部署架构图}可以得到软件整体的通信流程和部署思路,根据软件分层架构图\ref{fig:软件分层架构图}可以清晰地划分出功能模块,更高效地进行后续软件的开发。

\section{数据库设计}\label{sec:database}

本软件数据库设计分为三个阶段:概念模型设计、逻辑模型设计和物理模型设计,下面进行具体阐述。

\subsection{概念模型设计}

农业果实称重云端软件中,可以抽象出果实、称重记录、电子秤、采摘作业、用户、待办记录这六个实体对象,结合实际的业务场景,可以得到实体关系如表\ref{tab:ertab}所示。

\begin{table}[H]
\centering
\caption{实体关系表}
\label{tab:ertab}
\begin{tabular}{|c|c|c|}
\hline
关系名称 & 关系类型 & 描述 \\ \hline

采摘目标 & 1:N & 果实 1,采摘作业 N \\ \hline

称重目标 & 1:N & 果实 1,称重记录 N \\ \hline

称重人员 & 1:N & 员工 1,称重记录 N  \\ \hline

记录来源 & 1:N & 电子秤 1,称重记录 N \\ \hline

待办人员 & 1:N & 员工 1,待办记录 N \\ \hline

待办来源 & 1:N & 电子秤 1,待办记录 N \\ \hline

\end{tabular}%
\end{table}

将上述分析进行归纳,得到 ERD(Entity-Relationship Diagram, 即实体关系图)如图\ref{fig:ERD}所示。

\begin{figure}[H]
    \centering
    \includegraphics[width=0.8\linewidth]{../design/out/ERD.png}
    \caption{实体关系图}
    \label{fig:ERD}
\end{figure}

图\ref{fig:ERD}清晰地展示了各个实体之间的复杂关系,为后续进行数据库的逻辑模型设计奠定了基础。

\subsection{逻辑模型设计}

基于图\ref{fig:ERD}所展示的实体关系以及功能权限表\ref{tab:user_permissions}所展示的用户功能权限信息,下面阐述逻辑模型设计,描述系统各个表的结构及其字段定义,涵盖了表之间的关系、字段的数据类型及约束条件,如表\ref{tab:user},表\ref{tab:produce},表\ref{tab:work},表\ref{tab:scale},表\ref{tab:record},表\ref{tab:todo},表\ref{tab:mqttacl}所示。

% t_user
\begin{table}[H]
    \centering
    \caption{用户表 (t\_user)}
    \label{tab:user}
    \begin{tabular}{|l|l|l|l|}
    \hline
    字段 & 类型 & 约束 & 说明 \\
    \hline
    id & bigint & 主键 & 自增主键 \\
    uid & varchar(255) & 非空,唯一 & 用户认证编号 \\
    name & varchar(255) & 非空 & 用户名称 \\
    password & varchar(255) & 非空 & 密码(加密) \\
    roles & varchar(255) & - & 角色(英文逗号分隔) \\
    create\_time & bigint & 非空 & 创建时间(毫秒级时间戳) \\
    update\_time & bigint & 非空 & 更新时间(毫秒级时间戳) \\
    status & int & 非空 & 状态(0禁用/1启用/2已删除) \\
    \hline
    \end{tabular}
    \end{table}

表\ref{tab:user}中展现了软件中用户实体的具体字段设计,其中 roles 字段定义了用户角色,角色字段包含管理员(ROLE\_ADMIN)、采摘员工(ROLE\_EMPLOYEE)、电子秤(ROLE\_SCALE)以及系统用户(ROLE\_SYS),如果用户包含多个角色,则使用英文半角逗号隔开。

% t_produce
\begin{table}[H]
\centering
\caption{果实表 (t\_produce)}
\label{tab:produce}
\begin{tabular}{|l|l|l|l|}
\hline
字段 & 类型 & 约束 & 说明 \\
\hline
id & bigint & 主键 & 系统自带从0开始,用户添加从1000000开始 \\
name & varchar(255) & 非空,唯一 & 果实名称 \\
name\_en & varchar(255) & 唯一 & 果实英文名称 \\
create\_time & bigint & 非空 & 创建时间(毫秒级时间戳) \\
update\_time & bigint & 非空 & 更新时间(毫秒级时间戳) \\
status & int & 非空 & 状态(0未种植/1在种植/2已删除) \\
\hline
\end{tabular}
\end{table}

表\ref{tab:produce}中展现了软件中果实实体的具体字段设计。

% t_work
\begin{table}[H]
\centering
\caption{采摘作业表 (t\_work)}
\label{tab:work}
\begin{tabular}{|l|l|l|l|}
\hline
字段 & 类型 & 约束 & 说明 \\
\hline
id & bigint & 主键 & 自增主键 \\
produce\_id & bigint & 非空 & 果实编号 \\
start\_time & bigint & 非空 & 采摘开始时间(毫秒级时间戳) \\
end\_time & bigint & 非空 & 采摘结束时间(毫秒级时间戳) \\
data\_value & decimal(10,2) & - & 总的采摘称重结果 \\
unit & int & - & 称重单位(0mg/1g/2kg/3t/4lb/5oz/6ct) \\
create\_time & bigint & 非空 & 创建时间(毫秒级时间戳) \\
update\_time & bigint & 非空 & 更新时间(毫秒级时间戳) \\
status & int & 非空 & 状态(0未开始/1进行中/2已结束/3已取消/4已删除) \\
\hline
\end{tabular}
\end{table}

表\ref{tab:work}中展现了软件中采摘作业实体的具体字段设计。其中通过字段果实编号(produce\_id)关联了果实实体,字段对应果实表\ref{tab:produce}中的 id 字段。

% t_scale
\begin{table}[H]
\centering
\caption{电子秤表 (t\_scale)}
\label{tab:scale}
\begin{tabular}{|l|l|l|l|}
\hline
字段 & 类型 & 约束 & 说明 \\
\hline
id & bigint & 主键 & 自增主键 \\
sid & varchar(255) & 非空 & MQTT 用户认证编号 \\
model & varchar(255) & - & 型号 \\
max\_capacity & decimal(10,2) & 非空 & 最大量程 \\
min\_capacity & decimal(10,2) & 非空 & 最小量程 \\
unit & int & 非空 & 量程单位(0mg/1g/2kg/3t/4lb/5oz/6ct) \\
verification\_interval & int & - & 检定分度值 \\
display\_interval & int & - & 显示分度值 \\
unit\_dv & int & - & 分度值单位(0mg/1g/2kg/3t/4lb/5oz/6ct) \\
protocol & varchar(255) & - & 协议(小写,逗号分隔) \\
create\_time & bigint & 非空 & 创建时间(毫秒级时间戳) \\
update\_time & bigint & 非空 & 更新时间(毫秒级时间戳) \\
status & int & 非空 & 状态(0禁用/1启用/2已删除) \\
\hline
\end{tabular}
\end{table}

表\ref{tab:scale}中展现了软件中电子秤实体的具体字段设计。其中通过字段电子秤编号(sid)管理了用户实体,该字段对应于用户表\ref{tab:user}中的用户编号(uid)字段。

% t_record
\begin{table}[H]
\centering
\caption{称重记录表 (t\_record)}
\label{tab:record}
\begin{tabular}{|l|l|l|l|}
\hline
字段 & 类型 & 约束 & 说明 \\
\hline
id & bigint & 主键 & 自增主键 \\
work\_id & bigint & 非空 & 采摘作业编号 \\
produce\_id & bigint & 非空 & 果实编号 \\
image & varchar(255) & - & 果实图片地址 \\
employee\_id & bigint & 非空 & 员工编号 \\
scale\_id & bigint & 非空 & 电子秤编号 \\
data\_value & decimal(10,2) & 非空 & 称重结果 \\
data\_error\_margin & decimal(10,2) & - & 称量结果误差范围 \\
unit & int & 非空 & 称重单位(0mg/1g/2kg/3t/4lb/5oz/6ct) \\
data\_time & bigint & 非空 & 称重时间(毫秒级时间戳) \\
\hline
\end{tabular}
\end{table}

表\ref{tab:record}中展现了软件中称重记录实体的具体字段设计。通过字段果实编号(produce\_id)关联了果实实体,对应果实表\ref{tab:produce}中的 id 字段;通过字段采摘作业编号(work\_id)关联了作业实体,对应作业表\ref{tab:work}中的 id 字段;通过字段员工编号(employee\_id)关联了员工实体,对应用户表\ref{tab:user}中的 id 字段;通过字段电子秤编号(scale\_id)关联了电子秤实体,对应电子秤表\ref{tab:scale}中的 id 字段。

% t_todo
\begin{table}[H]
\centering
\caption{待处理称重记录表 (t\_todo)}
\label{tab:todo}
\begin{tabular}{|l|l|l|l|}
\hline
字段 & 类型 & 约束 & 说明 \\
\hline
id & bigint & 主键 & 自增主键 \\
produce\_id & bigint & - & 果实编号 \\
produce\_name & varchar(255) & - & 果实名称 \\
image & varchar(255) & - & 采摘作业图片地址 \\
employee\_id & bigint & 非空 & 员工编号 \\
scale\_id & bigint & 非空 & 电子秤编号 \\
data\_value & decimal(10,2) & 非空 & 称重结果 \\
data\_error\_margin & decimal(10,2) & - & 称量结果误差范围 \\
unit & int & 非空 & 称重单位(0mg/1g/2kg/3t/4lb/5oz/6ct) \\
data\_time & bigint & 非空 & 称重时间(毫秒级时间戳) \\
\hline
\end{tabular}
\end{table}

表\ref{tab:todo}中展现了软件中待办记录实体的具体字段设计。通过字段果实编号(produce\_id)关联了果实实体,对应果实表\ref{tab:produce}中的 id 字段;通过字段员工编号(employee\_id)关联了员工实体,对应用户表\ref{tab:user}中的 id 字段;通过字段电子秤编号(scale\_id)关联了电子秤实体,对应电子秤表\ref{tab:scale}中的 id 字段。

% t_mqtt_acl
\begin{table}[H]
\centering
\caption{MQTT 访问控制表 (t\_mqtt\_acl)}
\label{tab:mqttacl}
\begin{tabular}{|l|l|l|l|}
\hline
字段 & 类型 & 约束 & 说明 \\
\hline
id & int & 主键 & 自增主键 \\
username & varchar(255) & 非空 & MQTT 认证用户名 \\
permission & varchar(255) & 非空 & 操作权限(allow/deny,即允许/拒绝) \\
action & varchar(255) & 非空 & 操作类型(publish/subscribe/all,即发布/订阅/所有) \\
topic & varchar(255) & 非空 & 适用主题 \\
qos & int & - & 消息 QoS 级别(0,1,2) \\
retain & int & - & 是否支持发布保留消息(0否/1是) \\
\hline
\end{tabular}
\end{table}

表\ref{tab:mqttacl}中展现了软件中 MQTT 服务的访问控制条目的具体字段设计。通过字段 MQTT 认证用户名(username)管理了用户实体,对应用户表\ref{tab:user}中的 uid 字段。

数据库中每个表的设计都遵循规范化原则,以确保数据的一致性、完整性和高效性。字段的定义包括主键、唯一约束、非空约束等,考虑到更新数据的效率问题,取消了外键的设置,在后台服务中做相关数据更新的限制以确保数据一致性。

\subsection{物理模型设计}

软件使用 MySQL 作为数据库管理系统,采用 InnoDB 存储引擎,将事务隔离级别设置为可重复读,以确保数据的一致性和可靠性。下面从索引设计、数据存储与分布、数据访问控制和事务日志与恢复策略四个方面阐述物理模型设计。

索引设计方面:针对实际业务场景,基于逻辑模型设计中的表信息,设置了相关索引以提高查询更新效率,如表\ref{tab:index}所示。

% INDEX
\begin{table}[H]
\centering
\caption{索引设计表 (INDEX)}
\label{tab:index}
\begin{tabular}{|l|l|l|}
\hline
表 & 索引字段 & 说明 \\ \hline

t\_record & scale\_id & 高效查找电子秤历史称重记录 \\ \hline

t\_record & employee\_id & 高效查找员工历史称重记录 \\ \hline

t\_record & employee\_id, work\_id & 高效查找员工各采摘作业批次历史称重记录 \\ \hline

t\_record & work\_id & 高效查找各采摘作业批次历史称重记录 \\ \hline

t\_work & produce\_id & 高效查找果实对应的采摘作业 \\ \hline
\end{tabular}
\end{table}

数据存储与分布方面:采用主从架构,在靠近农场边端部署从库,在远端服务器部署主库,主从库采用\ref{sec:gtid}中提到的基于 GTID 的主从复制技术完成数据实时同步。

数据访问控制方面:主库允许读写,而从库只允许读,屏蔽任何用户的写操作,确保数据一致性。

事务日志与恢复策略方面:主从库均开启 BinLog(二进制日志)。在上述数据复制技术中,从库通过读取主库的二进制日志来同步数据。除此之外,在数据丢失或损坏的情况下,通过回放二进制日志来恢复丢失的数据。

\section{核心功能设计}\label{sec:keyFunction}

称重数据处理作为软件中的核心功能,涉及到多端数据处理、图像识别以及 MQTT 消息的配置等多个操作,这里对其进行更为详细的设计与分析。

称重数据的处理可分为边端处理和远端处理。如边端称重数据处理流程活动图\ref{fig:边端称重数据处理流程活动图}和远端称重数据处理流程活动图\ref{fig:远端称重数据处理流程活动图}所示。

\begin{figure}[H]
    \centering
    \includegraphics[width=0.8\linewidth]{../design/out/边端称重数据处理流程活动图.png}
    \caption{边端称重数据处理流程活动图}
    \label{fig:边端称重数据处理流程活动图}
\end{figure}

图\ref{fig:边端称重数据处理流程活动图}展现了以采摘员工、电子秤和边端服务为参与对象的活动泳道,描述了称重数据在边端向 MQTT Broker 的提交流程。其具体流程是:

1、采摘员工放置果实在电子秤上,然后刷员工卡;

2、电子秤对果实完成称重操作获取到称重结果,同时完成员工卡数据的读取,读取失败则返回错误信息,员工需要确认电子秤状态或者员工卡信息是否异常,然后选择重试或取消称重;

3、电子秤成功读取卡和称重数据后,拍摄果实图像并生成称重信息,开始尝试提交到边端服务的网关处;

4、边端服务的网关收到电子秤消息,首先完成协议转换,转换失败则返回错误信息,跳转到第 7 步;

5、转换成功后,进行认证授权操作。先进行认证,通过电子秤编号和密码完成认证,接着根据电子秤编号查询 MQTT 用户授权信息,确认电子秤是否有相关主题的发布功能。认证授权失败则返回错误信息,跳转到第 7 步;

6、认证授权成功后,即发布消息到 MQTT Broker,根据发布结果返回成功或失败信息;

7、如果提交成功或者重试次数超过电子秤设置的重试次数阈值,则退出第 3 步到第 7 步到循环,显示称重结果;

8、判断称重是否成功,如果成功则完成果实入库,否则返回第 1 步。

称重数据在边端完成提交并且称重消息在 MQTT Broker 处完成持久化后,便视为称重完成。接下来由监控 Broker 的远端服务对称重数据进一步处理,如远端称重数据处理流程活动图\ref{fig:远端称重数据处理流程活动图}所示。

\begin{figure}[H]
    \centering
    \includegraphics[width=0.8\linewidth]{../design/out/远端称重数据处理流程活动图.png}
    \caption{远端称重数据处理流程活动图}
    \label{fig:远端称重数据处理流程活动图}
\end{figure}

如图\ref{fig:远端称重数据处理流程活动图}所示,显示了远端接收相关称重消息后,称重消息在后台服务中的处理控制流程。具体步骤是:

1、远端服务的 MQTT 客户端接收到消息后,首先检查数据中员工和电子秤状态是否处于启用状态,如果不是则进入第 6 步;

2、根据图片识别果实,如果识别失败则进入第 6 步;

3、查询果实对应的采摘作业,如果不存在,则进入第 6 步;

4、检查作业和果实状态是否处于启用状态,如果不是则进入第 6 步;

5、前面步骤中没有发生异常,则持久化称重信息并发布处理成功的消息到 MQTT Broker,数据处理完成;

6、处理过程中发生异常,为防止数据丢失,将称重消息写入待办并发布处理失败的消息到 MQTT Broker。在管理后台待办界面中,由管理员人工完成异常称重数据的处理。

在最后称重数据处理成功或失败结果的消息发布到 MQTT Broker 之后,可以由特定的监控者订阅相关主题,进行告警或其它相关操作。除此之外,关于边端和远端称重数据在 MQTT Broker 的发布和订阅,还需要考虑 MQTT 消息的配置问题。对于消息主题的设置,发布/订阅称重消息和发布/订阅称重数据处理结果消息应该属于两个不同主题。对于传输质量级别的设置,需要确保消息有且仅被消费一次。根据\ref{sec:mqtt}中提到的三种 QoS 级别,这里应该选择 QoS2 级别,确保称重数据的可靠传输。

此外关于果实图像识别,通过调用果实图像服务所提供的接口完成果实图像识别。果实图像服务调用本地训练完成的模型进行推理,如果可信度超过 80\% 则视为识别成功,否则视为识别失败。

结合上述对远端和边端称重数据处理流程的设计和分析,可以作为代码编写的主要基础。

\section{服务接口设计}\label{sec:service}

本系统服务接口包含四个部分,分别是用户模块、果实模块、作业模块和称重模块。具体接口功能和路径设计如表\ref{tab:interface-user}、表\ref{tab:interface-produce}、表\ref{tab:interface-work}和表\ref{tab:interface-weigh}所示,接口风格均遵循 RESTful(Representational State Transfer, 即表现层状态转移)风格。

% 用户模块
\begin{table}[H]
\centering
\caption{用户模块接口设计}
\label{tab:interface-user}
\begin{tabular}{|c|c|c|}
\hline
接口名称 & 请求方法 & 接口路径 \\ \hline
用户获取个人信息 & GET & /user \\ \hline
管理员更新用户 & PUT & /user \\ \hline
管理员添加用户 & POST & /user \\ \hline
用户更新个人信息 & PUT & /user/me \\ \hline
用户登录 & POST & /user/login \\ \hline
管理员获取用户信息 & GET & /user/{uid} \\ \hline
管理员获取用户列表 & GET & /user/list \\ \hline
\end{tabular}
\end{table}

表\ref{tab:interface-user}显示了用户模块的接口设计,各接口以 /user 作为前缀。

% 果实模块
\begin{table}[H]
\centering
\caption{果实模块接口设计}
\label{tab:interface-produce}
\begin{tabular}{|c|c|c|}
\hline
接口名称 & 请求方法 & 接口路径 \\ \hline
果实图像推理识别 & POST & /predict \\ \hline
根据名称获取果实 & GET & /produce \\ \hline
更新果实 & PUT & /produce \\ \hline
添加果实 & POST & /produce \\ \hline
获取果实 & GET & /produce/{id} \\ \hline
获取果实年产量 & GET & /produce/summary/year \\ \hline
获取果实分批产量 & GET & /produce/summary/work \\ \hline
获取果实列表 & GET & /produce/list \\ \hline
\end{tabular}
\end{table}

表\ref{tab:interface-produce}显示了果实模块的接口设计,除了果实图像推理识别接口,其它各接口以 /produce 作为前缀。

% 作业模块
\begin{table}[H]
\centering
\caption{作业模块接口设计}
\label{tab:interface-work}
\begin{tabular}{|c|c|c|}
\hline
接口名称 & 请求方法 & 接口路径 \\ \hline
更新采摘作业 & PUT & /work \\ \hline
添加采摘作业 & POST & /work \\ \hline
获取采摘作业 & GET & /work/{id} \\ \hline
获取果实的采摘作业列表 & GET & /work/produce/{id}\\ \hline
获取采摘作业列表 & GET & /work/list \\ \hline
\end{tabular}
\end{table}

表\ref{tab:interface-work}显示了作业模块的接口设计,各接口以 /work 作为前缀。

% 称重模块
\begin{table}[H]
\centering
\caption{称重模块接口设计}
\label{tab:interface-weigh}
\begin{tabular}{|c|c|c|}
\hline
接口名称 & 请求方法 & 接口路径 \\\hline
更新电子秤信息 & PUT & /weigh/scale \\ \hline
添加电子秤 & POST & /weigh/scale \\\hline
添加称重记录 & POST & /weigh/record \\\hline
提交待处理称重记录 & POST & /weigh/record/todo \\\hline
获取待处理称重记录 & GET & /weigh/record/todo/list \\\hline
获取称重记录 & POST & /weigh/record/list \\\hline
添加称重结果监控者 & POST & /weigh/monitor \\\hline
获取员工各作业采摘量 & GET & /weigh/summary \\\hline
获取电子秤 & GET & /weigh/scale/{id} \\\hline
获取电子秤列表 & GET & /weigh/scale/list \\\hline
\end{tabular}
\end{table}

表\ref{tab:interface-weigh}显示了称重模块的接口设计,各接口以 /weigh 作为前缀。

根据本节提到所设计的服务接口,结合前面章节中提到的数据库设计和核心功能设计,可以为后续的系统实现提供规范化的接口调用方式。

\section{本章小结}

本章从软件架构设计、数据库设计、核心功能设计以及服务接口设计四个方面对农业果实称重云端软件进行了详细阐述。在软件架构设计中,明确了边端与远端协同的整体架构模式,并通过分层架构图清晰划分了各功能模块。在数据库设计中,从概念模型、逻辑模型到物理模型,逐步完善了数据库的设计方案,确保数据的一致性和高效性。在核心功能设计中,详细分析了称重数据处理的边端与远端流程,为后续实现提供了指导。在服务接口设计中,基于 RESTful 风格定义了各模块的接口,为系统的开发和集成提供了规范。本章内容为后续的系统实现和优化奠定了坚实的基础。
