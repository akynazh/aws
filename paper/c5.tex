\chapter{软件实现与测试}

本章将从三个方面展示软件的实现与测试结果,分别是功能需求测试结果、非功能需求测试结果以及果实图像识别模型的训练结果。

\section{软件功能需求测试结果}\label{sec:test-func}

1、测试环境:一台 8G 内存、8核 CPU 的 MacOS 系统。

2、部署方式:按照图\ref{fig:软件部署架构图}中的部署架构,使用 Docker Compose 技术组织各个应用容器,在本地完成部署和运行,如表\ref{tab:docker-compose}所示。

3、软件成果:远端服务、边端服务、后台管理界面、电子秤模拟器。

\begin{longtable}[ht]{|c|p{8cm}|c|}
\caption{Docker Compose 应用容器组织情况}
\label{tab:docker-compose}
\\
\hline
应用容器 & 容器描述 \\\hline
mysql-server & 远端 MySQL 数据库服务  \\\hline
mysql-edge & 边端 MySQL 数据库服务  \\\hline
redis & 远端 Redis 缓存服务  \\\hline
emqx1/emqx2/emqx3 & 边端 Emqx 静态集群  \\\hline
minio-remote & 远端 Minio 对象存储服务  \\\hline
minio-edge & 边端 Minio 对象存储服务  \\\hline
minio-bucket-init & 远端 Minio 客户端初始化服务  \\\hline
aws-edge & 边端 Spring Web 应用服务  \\\hline
aws-server & 远端 Spring Web 应用服务  \\\hline
aws-img & 远/边端 YOLO 图像识别服务  \\\hline
aws-web & 远端 Vue 后台管理界面  \\\hline
\end{longtable}

下面将首先给出总体的测试结果,然后详细展示各个核心功能的测试结果。

根据功能需求汇总表\ref{tab:req-summary}中提到的功能,给出对应的测试结果,如表\ref{tab:test-req-summary}所示。

\newpage
\begin{longtable}[ht]{|c|p{8cm}|c|}
\caption{功能需求测试结果}
\label{tab:test-req-summary}
\\
\hline
测试项 & 描述 & 测试结果 \\\hline
电子秤管理 & 验证能否添加、编辑、查看电子秤 & 测试通过 \\\hline
果实管理 & 验证能否添加、编辑、查看果实 & 测试通过 \\\hline
作业管理 & 验证能否添加、编辑、查看作业 & 测试通过 \\\hline
用户管理 & 验证能否添加、查看用户 & 测试通过 \\\hline
称重记录处理 & 验证能否处理称重数据 & 测试通过 \\\hline
待办记录处理 & 验证能否处理待办数据 & 测试通过 \\\hline
称重历史获取 & 验证能否查看个人和电子秤称重历史 & 测试通过 \\\hline
个人称重统计 & 验证能否查看和导出个人分作业批次采摘情况 & 测试通过 \\\hline
果实称重统计 & 验证能否查看和导出果实分作业批次采摘情况和年产量情况 & 测试通过 \\\hline
用户认证授权 & 验证能否认证和授权用户 & 测试通过 \\\hline
果实图像识别 & 验证识别果实图像 & 测试通过 \\\hline
果实图像存储 & 验证能否存储果实图像 & 测试通过  \\\hline
远端和边端的数据同步 & 验证能否同步远端和边端数据 & 测试通过  \\\hline
\end{longtable}

下面,针对\ref{sec:req1}给出核心需求用例分别进行详细的用例测试。

对提交称重数据需求用例\ref{tab:uc-weigh-submit}设计测试用例并进行测试,如表\ref{tab:uc-weigh-submit-test}所示。

\begin{longtable}[ht]{|c|p{8cm}|}
\caption{提交称重数据测试用例}
\label{tab:uc-weigh-submit-test}
\\
\hline
用例名称 & 提交称重数据 \\
\hline
用例描述 & 采摘员工在电子秤提交称重数据 \\
\hline
用例入口 & 电子秤提交称重数据 \\
\hline
测试步骤 & 1 运行电子秤模拟器 \\
& 2 点击生成数据 \\
& 3 点击识别果实 \\
& 4 点击提交数据 \\
\hline
测试用例 & 用例1: 通过 MQTT 协议提交正常数据 \\
& 用例2: 通过 CoAP 协议提交正常数据 \\
& 用例3: 通过 STOMP 协议提交正常数据 \\
& 用例4: 通过 HTTP 协议提交正常数据 \\
& 用例5: 通过 MQTT 协议提交异常数据(不存在的电子秤, 编号为999) \\
\hline
& 用例6: 通过 MQTT 协议提交异常数据(未启用的电子秤, 电子秤编号为1) \\
& 用例7: 通过 MQTT 协议提交异常数据(不存在的果实, 果实编号为999) \\
& 用例8: 通过 MQTT 协议提交异常数据(未启用的果实, 果实编号为1) \\
& 用例9: 通过 MQTT 协议提交异常数据(不存在的采摘作业,果实编号为3) \\
& 用例10: 通过 MQTT 协议提交异常数据(未开始或已结束的采摘作业,作业编号为4) \\
\hline
预期结果与实际结果 & 用例1:预期返回"成功",实际结果一致 \\
& 用例2:预期返回"成功",实际结果一致 \\
& 用例3:预期返回"成功",实际结果一致 \\
& 用例4:预期返回"成功",实际结果一致 \\
& 用例5:预期返回"成功",实际结果一致 \\
& 用例6:预期返回"失败",实际结果一致 \\
& 用例7:预期返回"失败",实际结果一致 \\
& 用例8:预期返回"失败",实际结果一致 \\
& 用例9:预期返回"失败",实际结果一致 \\
& 用例10:预期返回"失败",实际结果一致 \\
\hline
\end{longtable}

实际界面的测试结果如图\ref{fig:weigh-submit-result}所示。图中显示的是电子秤模拟器界面,包含9个输入框、4个操作按钮和1个结果显示区。

其中输入框包含通信协议(MQTT/HTTP/STOMP/CoAP)、电子秤编号、员工编号、果实编号、果实名称、果实图片、重量值、误差值和单位;提交按钮包含识别果实按钮、生成数据按钮、提交数据按钮和清除结果按钮;结果显示区包含记录编号、通信协议、作业编号、果实编号和提交结果。

\begin{figure}[H]
    \centering
    \includegraphics[width=0.8\linewidth]{../result/weigh-submit-result.png}
    \caption{提交称重数据用例测试截图}
    \label{fig:weigh-submit-result}
\end{figure}

对处理待办记录需求用例\ref{tab:uc-todo-handle}设计测试用例并进行测试,如表\ref{tab:uc-todo-handle-test}所示。

\begin{longtable}[ht]{|c|p{8cm}|}
\caption{处理待办记录测试用例}
\label{tab:uc-todo-handle-test}
\\
\hline
用例名称 & 处理待办记录 \\
\hline
用例描述 & 管理员在管理后台界面处理待办记录 \\
\hline
用例入口 & 后台管理界面中的待办管理模块 \\
\hline
测试步骤 & 1 点击提交待办 \\
& 2 选择果实种类 \\
& 3 点击确定,完成待办 \\
\hline
测试用例 & 用例1: 提交正常数据 \\
& 用例2: 提交异常数据(不存在的果实, 果实编号为999) \\
& 用例3: 提交异常数据(未启用的果实, 果实编号为1) \\
\hline
预期结果与实际结果 & 用例1:预期返回"成功",实际结果一致 \\
& 用例2:预期返回"失败",实际结果一致 \\
& 用例3:预期返回"失败",实际结果一致 \\
\hline
\end{longtable}

实际界面的测试情况如图\ref{fig:todo-handle-result-1}和图\ref{fig:todo-handle-result-2}所示。

\begin{figure}[H]
    \centering
    \includegraphics[width=0.8\linewidth]{../result/todo-handle-result-1.png}
    \caption{处理待办记录用例测试截图一}
    \label{fig:todo-handle-result-1}
\end{figure}

如图\ref{fig:todo-handle-result-1}所示,展现了软件管理后台中的待办管理界面,其中包含待办列表、4个操作按钮和分页块,其中待办列表表头包含编号、果实编号、果实名称、果实图像、员工编号、电子秤编号、称重数据、误差、称重时间和操作;操作按钮包含提交、丢弃、批量提交和批量丢弃。对某个待办项,点击提交按钮后,显示表单如图\ref{fig:todo-handle-result-2}所示。

\begin{figure}[H]
    \centering
    \includegraphics[width=0.8\linewidth]{../result/todo-handle-result-2.png}
    \caption{处理待办记录用例测试截图二}
    \label{fig:todo-handle-result-2}
\end{figure}

图\ref{fig:todo-handle-result-2}展现了待办项的提交表单,包含2个操作按钮和9个表单项,其中操作按钮包含取消和确认按钮;表单项包含选择果实、果实名称、果实编号、果实图像、员工编号、电子秤编号、称重数据、误差、称重时间,其中只能编辑果实,从选择果实项的下滑表单中选择果实后,果实名称和果实编号项将会随之变化。

对称重记录统计分析需求用例\ref{tab:uc-record-analysis}设计测试用例并进行测试,如表\ref{tab:uc-record-analysis-test}所示。

\begin{longtable}[ht]{|c|p{8cm}|}
\caption{称重记录统计分析测试用例}
\label{tab:uc-record-analysis-test}
\\
\hline
用例名称 & 称重记录统计分析 \\
\hline
用例描述 & 在管理后台界面查看果实和采摘人员的称重记录统计情况 \\
\hline
用例入口 & 后台管理界面中的果实/用户管理模块 \\
\hline
测试步骤 & 1 管理员访问果实/用户管理界面 \\
& 2 管理员点击某个可视化选项进行查看并导出 \\
\hline
\newpage
\hline
测试用例 & 用例1:点击查看苹果的年产量并导出为 Excel \\
& 用例2: 点击查看苹果的分批产量并导出为 Excel \\
& 用例3: 点击查看用户(用户编号为3)的分批产量并导出为 Excel  \\
\hline
预期结果与实际结果 & 用例1:预期得到折线图和对应的 Excel 表格,实际结果一致 \\
& 用例2:预期得到折线图和对应的 Excel 表格,实际结果一致 \\
& 用例3:预期得到折线图和对应的 Excel 表格,实际结果一致 \\
\hline
\end{longtable}

实际界面的测试情况如图\ref{fig:record-analysis-result-1}和图\ref{fig:record-analysis-result-2}所示。

\begin{figure}[H]
    \centering
    \includegraphics[width=0.8\linewidth]{../result/record-analysis-result-1.png}
    \caption{称重记录统计分析用例测试截图一}
    \label{fig:record-analysis-result-1}
\end{figure}

如图\ref{fig:record-analysis-result-1}所示,从左至右分别展现了果实的年产量、果实的分批产量和用户的分批采摘量和情况。其中果实的年产量折线图中,横坐标为年份,纵坐标为产量(kg);果实的分批产量折线图中,横坐标为作业编号,纵坐标为产量(kg);用户分批采摘量表格中,表头包含作业编号、果实名称和采摘量。

每个可视化图表都支持导出,点击图表右上方的导出按钮即可完成 Excel 导出,导出情况如图\ref{fig:record-analysis-result-2}所示。

\begin{figure}[H]
    \centering
    \includegraphics[width=0.8\linewidth]{../result/record-analysis-result-2.png}
    \caption{称重记录统计分析用例测试截图二}
    \label{fig:record-analysis-result-2}
\end{figure}

如图\ref{fig:record-analysis-result-2}所示,从左至右分别展现了果实的年产量表格、果实的分批产量表格和用户的分批采摘量表格。

对用户认证和授权需求用例\ref{tab:uc-user-auth}设计测试用例并进行测试,如表\ref{tab:uc-user-auth-test}所示。

\begin{longtable}[ht]{|c|p{8cm}|}
\caption{用户认证和授权测试用例}
\label{tab:uc-user-auth-test}
\\
\hline
用例名称 & 用户认证和授权 \\
\hline
用例描述 & 用户在后台管理界面中执行操作时的认证和授权流程 \\
\hline
用例入口 & 接口请求工具 Apifox \\
\hline
测试步骤 & 1 调用登录接口获取到认证令牌 \\
& 2 调用普通用户功能接口 \\
& 3 调用管理员功能接口 \\
\hline
测试用例 & 用例1:采摘员工调用查看个人信息接口 \\
& 用例2: 管理员调用查看个人信息接口 \\
& 用例3: 采摘员工调用查看用户列表接口 \\
& 用例4: 管理员调用查看用户列表接口 \\
\hline
预期结果与实际结果 & 用例1:预期返回数据和状态码 200,实际结果一致 \\
& 用例2:预期返回数据和状态码 200,实际结果一致 \\
\hline
& 用例3:预期返回错误信息和状态码 403,实际结果一致 \\
& 用例4:预期返回数据和状态码 200,实际结果一致 \\
\hline
\end{longtable}

实际界面的测试情况如图\ref{fig:user-auth-apifox}所示。

\begin{figure}[H]
    \centering
    \includegraphics[width=0.8\linewidth]{../result/user-auth-apifox.png}
    \caption{用户认证和授权用例测试截图}
    \label{fig:user-auth-apifox}
\end{figure}

如图\ref{fig:user-auth-apifox}所示,左侧展现了管理员调用接口的结果,左上调用查看个人信息接口,返回个人信息,状态码为 200,左下调用查看用户列表接口,返回用户列表,状态码为200;右侧展现了采摘员工调用接口的结果,左上调用查看个人信息接口,返回个人信息,状态码为 200,左下调用查看用户列表接口,返回 Access Denied(访问禁止),状态码为 403。

对果实图像识别需求用例\ref{tab:uc-produce-predict}设计测试用例并进行测试,如表\ref{tab:uc-produce-predict-test}所示。

\begin{longtable}[ht]{|c|p{8cm}|}
\caption{果实图像识别测试用例}
\label{tab:uc-produce-predict-test}
\\
\hline
用例名称 & 果实图像识别用例 \\
\hline
用例描述 & 调用本地模型推理预测果实种类 \\
\hline
用例入口 & 接口请求工具 Apifox \\
\hline
测试步骤 & 1 输入果实图片地址 \\
& 2 调用果实图像识别接口 \\
\hline
测试用例 & 用例1:提交模型支持的果实图片(苹果) \\
& 用例2: 提交模型不支持的果实图片(火龙果) \\
\hline
预期结果与实际结果 & 用例1:预期为成功预测果实种类为苹果且置信度高于 0.8,实际结果一致 \\
& 用例2:预期为错误预测果实种类且置信度低于 0.8,实际结果一致 \\
\hline
\end{longtable}

实际界面的测试情况如图\ref{fig:produce-predict-apifox}所示。

\begin{figure}[H]
    \centering
    \includegraphics[width=0.8\linewidth]{../result/produce-predict-apifox.png}
    \caption{果实图像识别用例测试截图}
    \label{fig:produce-predict-apifox}
\end{figure}

如图\ref{fig:produce-predict-apifox}所示,左侧提交了苹果图片(apple.png),返回结果中得到预测置信度排名第一位的果实为苹果,置信度为 0.93563,符合预期;右侧提交了火龙果图片(pitaya.jpeg),返回结果中得到预测置信度排名第一位的果实为甜土豆,置信度为 0.41996,低于预设值 0.8,说明预测失败,结果符合预期。

对新建采摘作业需求用例\ref{tab:uc-work-new}设计测试用例并进行测试,如表\ref{tab:uc-work-new-test}所示。

\begin{longtable}[ht]{|c|p{8cm}|}
\caption{新建采摘作业测试用例}
\label{tab:uc-work-new-test}
\\
\hline
用例名称 & 新建采摘作业用例 \\
\hline
用例描述 & 对某一果实新建对于采摘作业 \\
\hline
用例入口 & 后台管理界面中的作业管理模块 \\
\hline
测试步骤 & 1 点击添加作业 \\
& 2 选择采摘产品 \\
& 3 选择起止时间 \\
& 4 点击确认完成添加 \\
\hline
测试用例 & 用例1: 提交正常数据(选择已启用的果实:苹果) \\
& 用例2: 提交异常数据(选择未启用的果实:香蕉) \\
\hline
& 用例3: 提交异常数据(选择已有对应采摘作业的果实:苹果) \\
\hline
预期结果与实际结果 & 用例1:预期返回"成功",实际结果一致 \\
& 用例2:预期返回"失败",实际结果一致 \\
& 用例3:预期返回"失败",实际结果一致 \\
\hline
\end{longtable}

实际界面的测试情况如图\ref{fig:work-new-result-1}和图\ref{fig:work-new-result-2}所示。

\begin{figure}[H]
    \centering
    \includegraphics[width=0.8\linewidth]{../result/work-new-result-1.png}
    \caption{新建采摘作业用例测试截图一}
    \label{fig:work-new-result-1}
\end{figure}

如图\ref{fig:work-new-result-1}所示,展现了软件管理后台中的作业管理界面,其中包含作业列表、4个操作按钮和分页块,其中作业列表表头包含编号、采摘产品、开始时间、结束时间、采摘量、状态和操作;操作按钮包含添加作业、查询作业、导出作业数据和编辑。点击添加作业按钮后,显示表单如图\ref{fig:work-new-result-2}所示。

\begin{figure}[H]
    \centering
    \includegraphics[width=0.8\linewidth]{../result/work-new-result-2.png}
    \caption{新建采摘作业用例测试截图二}
    \label{fig:work-new-result-2}
\end{figure}

图\ref{fig:work-new-result-2}展现了作业项的提交表单,包含2个操作按钮和3个表单项,其中操作按钮包含取消和确认按钮;表单项包含选择采摘产品、开始时间和结束时间,采摘产品在下滑表单中选择,开始时间和结束时间点击后在日历组件中选择。

至此,完成对核心功能需求用例的测试。

\section{软件非功能需求测试结果}

根据非功能需求汇总表\ref{tab:n-req-summary}中提到的功能,给出对应的实现结果,如表\ref{tab:test-n-req-summary}所示。

\begin{longtable}[ht]{|c|p{8cm}|c|}
\caption{非功能需求实现结果}
\label{tab:test-n-req-summary}
\\
\hline
测试项 & 描述 & 测试结果 \\\hline
支持多种协议 & 支持 MQTT、HTTP、CoAP、STOMP 四种协议 & 验证通过 \\\hline
支持多种提交方式 & 支持通过果实图像/名称/编号提交称重数据 & 验证通过 \\\hline
支持模拟提交流程 & 支持通过电子秤模拟器模拟数据提交 & 验证通过 \\\hline
支持受限网络环境 & 支持在边端提交数据 & 验证通过 \\\hline
提供高性能的服务 & 提供高性能的果实图像识别服务和称重数据提交服务 & 验证通过 \\\hline
提供安全性保障 & 数据库加密敏感字段,所有接口提供认证授权保护 & 验证通过 \\\hline
\end{longtable}

针对提供高性能服务这个非功能需求,下面给出具体的测试和分析结果。首先给出对果实图像识别服务的性能测试,然后再给出称重数据提交服务的性能测试。

软件基于 YOLOv8 实现果实图像识别功能,训练数据集来源于 Kaggle 平台上的已经完成数据标注的 Fruits-360 数据集,从该数据集中挑选了来自 22 种不同果实的近两千张照片,将 70\%的数据作为训练集,20\%的数据作为验证集,10\%的数据作为测试集,使用 yolov8m 预训练模型作为基础,在单个 GPU P100 上训练了 50 个 epoch(训练轮次)。

下面对训练出来的模型进行性能测试:

1、测试环境:4 核 CPU, 31.4GB 内存, 2 个 Tesla T4 GPU。

2、测试方式:使用 Ultralytics 提供的性能测试库,调用训练出来的最佳模型 best.pt,测试数据则采用训练集数据。

3、测试结果:如图 \ref{fig:model-benchmark} 所示。

\begin{figure}[H]
    \centering
    \includegraphics[width=0.8\linewidth]{../source/aws-img/yolov8/benchmark.png}
    \caption{果实图像识别模型性能测试结果}
    \label{fig:model-benchmark}
\end{figure}

从图 \ref{fig:model-benchmark} 可以得出,模型使用 PyTorch 框架训练,推理时间为 19.42 毫秒,帧率为 51.49 FPS,且在 mAP50-95 指标下的得分为 0.5823,适用于精度和推理速度之间的平衡。

此外,训练模型得到的标准化混淆矩阵图如下:

\begin{figure}[H]
    \centering
    \includegraphics[width=0.8\linewidth]{../source/aws-img/yolov8/out/image/confusion_matrix_normalized.png}
    \caption{果实图像识别模型训练结果-标准化混淆矩阵图}
    \label{fig:confusion_matrix_normalized}
\end{figure}

从图\ref{fig:confusion_matrix_normalized}中的标准化混淆矩阵可以看出,模型在多个类别上表现出色。例如,apple(苹果)、banana(香蕉) 和 carrot(胡萝卜) 等类别的预测准确率都超过了 90\%。然而,某些类别的分类效果较差,chilli\_pepper(辣椒)的准确率为 62\%,并且容易被误分类为 corn(玉米) 或其他类别。总体来说,模型在多数类别上的表现良好,大部分类别的置信度都在 0.7 到 1.0 之间。

综上所述,训练出来的果实图像识别模型在精度与推理速度之间达到平衡,模型在多数类别上的识别效果表现良好。

对于称重数据的提交,下面根据压力测试的方法\cite{Zhu2017}来对其进行性能测试,通过测试软件 JMeter 模拟 HTTP 并发请求来统计各项性能指标:

1、测试环境:4 核 CPU, 4GB 内存 的 Docker 容器。

2、测试方式:启动称重相关服务,包含 Spring 应用服务、EMQX 集群以及MySQL 数据库服务,分别以并发数 500、1000 和 2000 对提交称重数据接口进行测试,Ramp-up 时间(负载增长时间) 设为 30s,通信协议采用 HTTP。

3、测试结果:如表\ref{tab:jmeter-test}所示。

\begin{longtable}[ht]{|p{1.5cm}|p{2cm}|p{2cm}|p{2cm}|p{2cm}|p{1.5cm}|}
\caption{核心接口的压力测试结果}
\label{tab:jmeter-test}
\\\hline
并发数 & 平均响应时间(ms) & 最小响应时间(ms) & 最大响应时间(ms) & 90\%响应时间(ms) & 吞吐量 \\\hline
500 & 608 & 55 & 1698 & 1033 & 166.2 \\\hline
1000 & 881 & 150 & 2028 & 1441 & 350 \\\hline
2000 & 1376 & 89 & 3328 & 2223 & 384.5 \\\hline
\end{longtable}

对与中小型的农场,日常状态下的并发压力并不大,而在优惠活动时的并发
可能达到上百上千,由表 5.9 知道,在查询性能方面,并发数 1000 以内时,电商系统的查询时间基本在 2 秒以内;在并发数 2000 以内,电商系统的查询时间的平均值约1.4 秒,90\%的响应在约2.2 秒内,最长时间约3.3 秒,对于中小企业电商,绝大部分通常时间内系统可以良好的体验,当用户的并发请求过高时,超过 2000,可能会有较高的延迟,活动期间可能需要扩容和异步削峰处理。

\section{本章小结}

TODO