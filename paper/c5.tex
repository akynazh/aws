\chapter{软件实现与测试}

本章将从三个方面展示软件的实现与测试结果,分别是功能需求测试结果、非功能需求测试结果以及果实图像识别模型的训练结果。

\section{软件功能需求测试结果}\label{sec:test-func}

1、测试环境:一台 8G 内存、8核 CPU 的 MacOS 系统。

2、部署方式:按照图\ref{fig:软件部署架构图}中的部署架构,使用 Docker Compose 技术组织各个应用容器,在本地完成部署和运行,如表\ref{tab:docker-compose}所示。

3、软件成果:远端服务、边端服务、后台管理界面、电子秤模拟器。

\begin{longtable}[ht]{|c|p{8cm}|c|}
\caption{Docker Compose 应用容器组织情况}
\label{tab:docker-compose}
\\
\hline
应用容器 & 容器描述 \\\hline
mysql-server & 远端 MySQL 数据库服务  \\\hline
mysql-edge & 边端 MySQL 数据库服务  \\\hline
redis & 远端 Redis 缓存服务  \\\hline
emqx1/emqx2/emqx3 & 边端 Emqx 静态集群  \\\hline
minio-remote & 远端 Minio 对象存储服务  \\\hline
minio-edge & 边端 Minio 对象存储服务  \\\hline
minio-bucket-init & 远端 Minio 客户端初始化服务  \\\hline
aws-edge & 边端 Spring Web 应用服务  \\\hline
aws-server & 远端 Spring Web 应用服务  \\\hline
aws-img & 远/边端 YOLO 图像识别服务  \\\hline
aws-web & 远端 Vue 后台管理界面  \\\hline
\end{longtable}

下面将首先给出总体的测试结果,然后详细展示各个核心功能的测试结果。

根据功能需求汇总表\ref{tab:req-summary}中提到的功能,给出对应的测试结果,如表\ref{tab:test-req-summary}所示。

\newpage
\begin{longtable}[ht]{|c|p{8cm}|c|}
\caption{功能需求测试结果}
\label{tab:test-req-summary}
\\
\hline
测试项 & 描述 & 测试结果 \\\hline
电子秤管理 & 验证能否添加、编辑、查看电子秤 & 测试通过 \\\hline
果实管理 & 验证能否添加、编辑、查看果实 & 测试通过 \\\hline
作业管理 & 验证能否添加、编辑、查看作业 & 测试通过 \\\hline
用户管理 & 验证能否添加、查看用户 & 测试通过 \\\hline
称重记录处理 & 验证能否处理称重数据 & 测试通过 \\\hline
待办记录处理 & 验证能否处理待办数据 & 测试通过 \\\hline
称重历史获取 & 验证能否查看个人和电子秤称重历史 & 测试通过 \\\hline
个人称重统计 & 验证能否查看和导出个人分作业批次采摘情况 & 测试通过 \\\hline
果实称重统计 & 验证能否查看和导出果实分作业批次采摘情况和年产量情况 & 测试通过 \\\hline
用户认证授权 & 验证能否认证和授权用户 & 测试通过 \\\hline
果实图像识别 & 验证识别果实图像 & 测试通过 \\\hline
果实图像存储 & 验证能否存储果实图像 & 测试通过  \\\hline
远端和边端的数据同步 & 验证能否同步远端和边端数据 & 测试通过  \\\hline
\end{longtable}

下面,针对\ref{sec:req1}给出核心需求用例分别进行详细的用例测试。

对提交称重数据需求用例\ref{tab:uc-weigh-submit}设计测试用例并进行测试,如表\ref{tab:uc-weigh-submit-test}所示。

\begin{longtable}[ht]{|c|p{8cm}|}
\caption{提交称重数据测试用例}
\label{tab:uc-weigh-submit-test}
\\
\hline
用例名称 & 提交称重数据 \\
\hline
用例描述 & 采摘员工在电子秤提交称重数据 \\
\hline
用例入口 & 电子秤提交称重数据 \\
\hline
测试步骤 & 1 运行电子秤模拟器 \\
& 2 点击生成数据 \\
& 3 点击识别果实 \\
& 4 点击提交数据 \\
\hline
测试用例 & 用例1: 通过 MQTT 协议提交正常数据 \\
& 用例2: 通过 CoAP 协议提交正常数据 \\
& 用例3: 通过 STOMP 协议提交正常数据 \\
& 用例4: 通过 HTTP 协议提交正常数据 \\
& 用例5: 通过 MQTT 协议提交异常数据(不存在的电子秤, 编号为999) \\
& 用例6: 通过 MQTT 协议提交异常数据(未启用的电子秤, 电子秤编号为1) \\
& 用例7: 通过 MQTT 协议提交异常数据(不存在的果实, 果实编号为999) \\
& 用例8: 通过 MQTT 协议提交异常数据(未启用的果实, 果实编号为1) \\
\hline
& 用例9: 通过 MQTT 协议提交异常数据(不存在的采摘作业,果实编号为3) \\
& 用例10: 通过 MQTT 协议提交异常数据(未开始或已结束的采摘作业,作业编号为4) \\
\hline
预期结果与实际结果 & 用例1:预期返回"成功",实际结果一致 \\
& 用例2:预期返回"成功",实际结果一致 \\
& 用例3:预期返回"成功",实际结果一致 \\
& 用例4:预期返回"成功",实际结果一致 \\
& 用例5:预期返回"成功",实际结果一致 \\
& 用例6:预期返回"失败",实际结果一致 \\
& 用例7:预期返回"失败",实际结果一致 \\
& 用例8:预期返回"失败",实际结果一致 \\
& 用例9:预期返回"失败",实际结果一致 \\
& 用例10:预期返回"失败",实际结果一致 \\
\hline
\end{longtable}

实际界面的测试结果如图\ref{fig:weigh-submit-result}所示。图中显示的是电子秤模拟器界面,包含9个输入框、4个操作按钮和1个结果显示区。

其中输入框包含通信协议(MQTT/HTTP/STOMP/CoAP)、电子秤编号、员工编号、果实编号、果实名称、果实图片、重量值、误差值和单位;提交按钮包含识别果实按钮、生成数据按钮、提交数据按钮和清除结果按钮;结果显示区包含记录编号、通信协议、作业编号、果实编号和提交结果。

\begin{figure}[H]
    \centering
    \includegraphics[width=0.8\linewidth]{../result/weigh-submit-result.png}
    \caption{提交称重数据用例测试截图}
    \label{fig:weigh-submit-result}
\end{figure}

对处理待办记录需求用例\ref{tab:uc-todo-handle}设计测试用例并进行测试,如表\ref{tab:uc-todo-handle-test}所示。

\begin{longtable}[ht]{|c|p{8cm}|}
\caption{处理待办记录测试用例}
\label{tab:uc-todo-handle-test}
\\
\hline
用例名称 & 处理待办记录 \\
\hline
用例描述 & 管理员在管理后台界面处理待办记录 \\
\hline
用例入口 & 后台管理界面中的待办管理模块 \\
\hline
测试步骤 & 1 点击提交待办 \\
& 2 选择果实种类 \\
& 3 点击确定,完成待办 \\
\hline
测试用例 & 用例1: 提交正常数据 \\
& 用例2: 提交异常数据(不存在的果实, 果实编号为999) \\
& 用例3: 提交异常数据(未启用的果实, 果实编号为1) \\
\hline
预期结果与实际结果 & 用例1:预期返回"成功",实际结果一致 \\
& 用例2:预期返回"失败",实际结果一致 \\
& 用例3:预期返回"失败",实际结果一致 \\
\hline
\end{longtable}

实际界面的测试情况如图\ref{fig:todo-handle-result-1}和图\ref{fig:todo-handle-result-2}所示。

\begin{figure}[H]
    \centering
    \includegraphics[width=0.8\linewidth]{../result/todo-handle-result-1.png}
    \caption{处理待办记录用例测试截图一}
    \label{fig:todo-handle-result-1}
\end{figure}

如图\ref{fig:todo-handle-result-1}所示,展现了软件管理后台中的待办管理界面,其中包含待办列表、4个操作按钮和分页块,其中待办列表表头包含编号、果实编号、果实名称、果实图像、员工编号、电子秤编号、称重数据、误差、称重时间和操作;操作按钮包含提交、丢弃、批量提交和批量丢弃。对某个待办项,点击提交按钮后,显示表单如图\ref{fig:todo-handle-result-2}所示。

\begin{figure}[H]
    \centering
    \includegraphics[width=0.8\linewidth]{../result/todo-handle-result-2.png}
    \caption{处理待办记录用例测试截图二}
    \label{fig:todo-handle-result-2}
\end{figure}

图\ref{fig:todo-handle-result-2}展现了待办项的提交表单,包含2个操作按钮和9个表单项,其中操作按钮包含取消和确认按钮;表单项包含选择果实、果实名称、果实编号、果实图像、员工编号、电子秤编号、称重数据、误差、称重时间,其中只能编辑果实,从选择果实项的下滑表单中选择果实后,果实名称和果实编号项将会随之变化。

对用户认证和授权需求用例\ref{tab:uc-user-auth}设计测试用例并进行测试,如表\ref{tab:uc-user-auth-test}所示。

\begin{longtable}[ht]{|c|p{8cm}|}
\caption{用户认证和授权测试用例}
\label{tab:uc-user-auth-test}
\\
\hline
用例名称 & 用户认证和授权 \\
\hline
用例描述 & 用户在后台管理界面中执行操作时的认证和授权流程 \\
\hline
用例入口 & 接口请求工具 Postman \\
\hline
测试步骤 & 1 调用登录接口获取到认证令牌 \\
& 2 调用普通用户功能接口 \\
& 3 调用管理员功能接口 \\
\hline
测试用例 & 用例1:采摘员工调用查看个人信息接口 \\
& 用例2: 管理员调用查看个人信息接口 \\
& 用例3: 采摘员工调用查看用户列表接口 \\
& 用例4: 管理员调用查看用户列表接口 \\
\hline
预期结果与实际结果 & 用例1:预期返回"成功",实际结果一致 \\
& 用例2:预期返回"成功",实际结果一致 \\
& 用例3:预期返回"失败",实际结果一致 \\
& 用例3:预期返回"成功",实际结果一致 \\
\hline
\end{longtable}

对果实图像识别需求用例\ref{tab:uc-produce-predict}设计测试用例并进行测试,如表\ref{tab:uc-produce-predict-test}所示。

\begin{longtable}[ht]{|c|p{8cm}|}
\caption{果实图像识别测试用例}
\label{tab:uc-produce-predict-test}
\\
\hline
用例名称 & 果实图像识别用例 \\
\hline
用例描述 & 调用本地模型推理预测果实种类 \\
\hline
用例入口 & 接口请求工具 Postman \\
\hline
测试步骤 & 1 输入果实图片地址 \\
& 2 调用果实图像识别接口 \\
\hline
测试用例 & 用例1:提交模型支持的果实图片(苹果) \\
& 用例2: 提交模型不支持的果实图片(火龙果) \\
\hline
预期结果与实际结果 & 用例1:预期返回"成功",实际结果一致 \\
& 用例2:预期返回"失败",实际结果一致 \\
\hline
\end{longtable}

对新建采摘作业需求用例\ref{tab:uc-work-new}设计测试用例并进行测试,如表\ref{tab:uc-work-new-test}所示。

\begin{longtable}[ht]{|c|p{8cm}|}
\caption{新建采摘作业测试用例}
\label{tab:uc-work-new-test}
\\
\hline
用例名称 & 新建采摘作业用例 \\
\hline
用例描述 & 对某一果实新建对于采摘作业 \\
\hline
用例入口 & 后台管理界面中的作业管理模块 \\
\hline
测试步骤 & 1 点击添加作业 \\
& 2 选择采摘产品 \\
& 3 选择起止时间 \\
& 4 点击确认完成添加 \\
\hline
测试用例 & 用例1: 提交正常数据(选择已启用的果实:苹果) \\
& 用例2: 提交异常数据(选择未启用的果实:香蕉) \\
& 用例3: 提交异常数据(选择已有对应采摘作业的果实:苹果) \\
\hline
预期结果与实际结果 & 用例1:预期返回"成功",实际结果一致 \\
& 用例2:预期返回"失败",实际结果一致 \\
& 用例3:预期返回"失败",实际结果一致 \\
\hline
\end{longtable}

实际界面的测试情况如图\ref{fig:work-new-result-1}和图\ref{fig:work-new-result-2}所示。

\begin{figure}[H]
    \centering
    \includegraphics[width=0.8\linewidth]{../result/work-new-result-1.png}
    \caption{新建采摘作业用例测试截图一}
    \label{fig:work-new-result-1}
\end{figure}

如图\ref{fig:work-new-result-1}所示,展现了软件管理后台中的作业管理界面,其中包含作业列表、4个操作按钮和分页块,其中作业列表表头包含编号、采摘产品、开始时间、结束时间、采摘量、状态和操作;操作按钮包含添加作业、查询作业、导出作业数据和编辑。点击添加作业按钮后,显示表单如图\ref{fig:work-new-result-2}所示。

\begin{figure}[H]
    \centering
    \includegraphics[width=0.8\linewidth]{../result/work-new-result-2.png}
    \caption{新建采摘作业用例测试截图二}
    \label{fig:work-new-result-2}
\end{figure}

图\ref{fig:work-new-result-2}展现了作业项的提交表单,包含2个操作按钮和3个表单项,其中操作按钮包含取消和确认按钮;表单项包含选择采摘产品、开始时间和结束时间,采摘产品在下滑表单中选择,开始时间和结束时间点击后在日历组件中选择。

至此,完成对核心功能需求用例的测试。

\section{软件非功能需求测试结果}

根据非功能需求汇总表\ref{tab:n-req-summary}中提到的功能,给出对应的实现结果,如表\ref{tab:test-n-req-summary}所示。

\begin{longtable}[ht]{|c|p{8cm}|c|}
\caption{非功能需求实现结果}
\label{tab:test-n-req-summary}
\\
\hline
测试项 & 描述 & 测试结果 \\\hline
支持多种协议 & 支持 MQTT、HTTP、CoAP、STOMP 四种协议 & 验证通过 \\\hline
支持多种提交方式 & 支持通过果实图像/名称/编号提交称重数据 & 验证通过 \\\hline
支持模拟提交流程 & 支持通过电子秤模拟器模拟数据提交 & 验证通过 \\\hline
支持受限网络环境 & 支持在边端提交数据 & 验证通过 \\\hline
提供高性能的接口 & 提高在单位时间内处理请求的能力,QPS(Queries Per Second, 每秒请求数) 需要大于 1000 & 验证通过 \\\hline
提供安全性保障 & 数据库加密敏感字段,所有接口提供认证授权保护 & 验证通过 \\\hline
\end{longtable}

针对核心接口的性能测试

1、测试环境
2、测试用例
3、测试结果

\section{果实图像识别模型的训练结果}\label{sec:test-model}

农业果实称重云端软件基于 YOLOv8 实现果实图像识别功能,训练数据集来源于 Kaggle 平台上的已经完成数据标注的 Fruits-360 数据集。训练流程如下:

1、挑选来自 22 种不同果实的近三千张照片,将其上传至 Roboflow 平台;

2、在 Kaggle 云平台上,通过 Roboflow 提供的 API,将数据导出为 YOLOv8 格式的数据集;

3、使用 YOLOv8 模型训练数据集。训练进行 50 个 epoch,使用 yolov8m 预训练模型作为基础。训练过程中,如果验证集的性能在 30 个 epoch 内没有改进,训练将提前停止。

模型训练完成后,得到标准化混淆矩阵图如下:

\begin{figure}[H]
    \centering
    \includegraphics[width=0.8\linewidth]{../source/aws-img/yolov8/out/image/confusion_matrix_normalized.png}
    \caption{果实图像识别模型训练结果-标准化混淆矩阵图}
    \label{fig:confusion_matrix_normalized}
\end{figure}

从图\ref{fig:confusion_matrix_normalized}中的标准化混淆矩阵可以看出,模型在多个类别上表现出色。例如,apple(苹果)、banana(香蕉) 和 carrot(胡萝卜) 等类别的预测准确率都超过了 90\%,表明模型在这些类别上的预测非常精确。

然而,某些类别的分类效果较差,尤其是在 chilli\_pepper(辣椒) 和 background(背景) 类别上。chilli\_pepper 的准确率为 62\%,并且容易被误分类为 corn(玉米) 或其他类别。

此外,模型在识别 background 类别时的准确率较低,为 18\%,这可能是由于背景图像的多样性和噪声所致。

总体来说,模型在多数类别上的表现良好,但对于某些类别的混淆仍然存在,尤其是那些外观相似的类别(如 bell\_pepper(甜椒) 和 cabbage(白菜))。这种混淆可能来源于数据集中的样本不均衡、类别相似性较高或特征不足等问题。

\section{本章小结}