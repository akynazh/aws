在现代农业向智慧化转型的进程中,精准收获成为提升农业生产效率与质量的关键环节。其中,果蔬采摘的称重工作是精准收获的重要组成部分。随着电子秤的广泛推广应用,其在智慧农场果蔬采摘中的作用日益凸显。然而,目前果蔬采摘很大程度上依旧依赖人工操作,这就使得对采摘果实进行准确的称重记录以及对果蔬产量进行科学管理变得尤为迫切。

为满足上述需求,本研究基于Java技术栈和Vue框架设计并实现了一款前后端分离的果实称重云端软件。在前端部分,选用Vue框架进行构建。Vue具有简洁的语法、高效的虚拟DOM机制以及良好的组件化开发模式,能够为用户提供流畅、美观且交互性强的操作界面,使用户可以方便地进行电子秤配置、数据查看与管理等操作。后端则采用Java技术栈,利用其丰富的类库和强大的跨平台性能,确保系统的稳定性和可扩展性。通过Spring Boot框架简化开发流程,使用Spring JPA进行数据库操作,保证系统能够高效处理大量的称重数据。

该软件具备多项核心功能。首先,支持基于HTTP、MQTT等不同协议的电子秤,能够兼容市场上多种类型的电子秤设备,增强了系统的通用性和适应性。其次,提供了灵活的配置功能,用户可以方便地新增电子秤,设置对应的秘钥,以及根据不同电子秤的数据特点配置数据格式,确保能够准确解析和处理来自各种电子秤的称重数据。此外,软件还具备强大的统计分析能力,可以按照批次、采摘人员等维度对称重数据进行深入分析,能够将分析结果导出为常见的数据格式,并自动生成详细的统计报告,为农场管理者提供决策依据。

为了验证软件的功能和性能,本研究模拟了称重终端与云端软件的交互过程。通过模拟不同协议的电子秤向云端软件发送称重数据,全面测试软件在数据接收、解析、存储和处理等方面的功能完整性。同时,从多个维度对软件性能进行评估,包括可用性、每秒查询率(QPS)、并发用户数和响应时间等。可用性方面,通过长时间运行测试,确保软件在各种情况下都能稳定可靠地运行;QPS测试用于评估软件在单位时间内处理查询请求的能力;并发用户数测试模拟多个用户同时使用软件的场景,检验系统在高并发情况下的性能表现;响应时间测试则关注用户操作后软件的响应速度,以保证良好的用户体验。

综上所述,本研究实现的果实称重云端软件为智慧农场的果蔬采摘称重管理提供了一个高效、稳定且功能丰富的解决方案,有助于推动智慧农场的精准收获和数字化管理。 