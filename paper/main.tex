\documentclass{xduugthesis}
\AtBeginDocument{\ctexset{subsection/format+=\centering}}
\xdusetup{
style = { 
    cjk-font = fandol, 
    latin-font = gyre,
    bib-backend = biblatex % -> biber
},
info = {
    title = {农业果实称重云端软件的设计与实现},
    department = {计算机科学与技术院},
    major = {计算机科学与技术},
    author = {江志航},
    supervisor = {霍秋艳},
    class-id = {2103018},
    student-id = {20009100359},
    abstract = {chapters/abstract-zh.tex},
    abstract* = {chapters/abstract-en.tex},
    keywords = {智慧农业, 称重系统, Spring Boot, Vue, MQTT, 信息管理系统},
    keywords* = {Smart agriculture, Weighing System, Spring Boot, Vue, MQTT, Information management system},
    bib-resource = {reference.bib},
}
}
\begin{document}

\chapter{绪论}

\section{研究背景与意义}

在全球人口持续增长以及人们对农产品品质与安全关注度日益提升的大背景下,农业的发展面临着前所未有的挑战与机遇。智慧农业作为一种融合了现代信息技术、生物技术、工程技术等多学科技术的新型农业发展模式\cite{赵春江2021智慧农业的发展现状与未来展望},正逐渐成为推动农业现代化进程的关键力量。它通过对农业生产过程中的各种数据进行实时采集、传输、分析和处理,实现了农业生产的精准化、智能化和自动化管理\cite{李道亮2012物联网与智慧农业},有效提高了农业生产效率、降低了生产成本、保障了农产品质量安全,为解决当前农业发展面临的诸多问题提供了新的思路和方法。

果实称重作为精准农业的重要环节,对于农业生产管理和决策具有不可或缺的作用\cite{罗锡文2016信息技术提升农业机械化水平}。准确的果实称重数据能够为农民提供有关农作物产量的精确信息,帮助他们及时了解农作物的生长状况\cite{翁杨2019基于深度学习的农业植物表型研究综述},进而合理安排农事活动,如施肥、灌溉、病虫害防治等,以提高农作物的产量和质量。通过对果实称重数据的分析,还可以为农产品的市场定价、销售策略制定以及供应链管理提供有力的数据支持,有助于优化农产品的流通环节,提高农业经济效益\cite{Lipcsei2021AnalysisOA}。在水果采摘环节,精确的称重能够实现水果按重量分级,满足不同消费者的需求,提升水果的市场竞争力\cite{Ji2019};在农产品加工行业,准确的原料称重是保证产品质量一致性和生产工艺稳定性的关键因素。

然而,传统的果实称重方式主要依赖人工操作,存在着诸多明显的不足。人工称重效率低下,在大规模的农业生产中,需要耗费大量的人力和时间,尤其是在果实收获季节,繁重的称重工作往往使人力成本大幅增加\cite{Jiang2012}。人工称重容易受到人为因素的影响,导致称重数据的准确性和可靠性难以保证,如操作人员的疲劳、疏忽以及读数误差等,都可能使称重数据出现偏差,从而影响后续的生产决策\cite{Chen2002}。同时,这种传统的称重方式在数据记录和管理上也较为繁琐,难以实现数据的实时共享和高效分析,无法满足现代精准农业对数据快速处理和深度挖掘的需求\cite{Widagdo2020RecordingSO}。

随着云计算、物联网、大数据等信息技术的飞速发展,将这些先进技术应用于农业领域,构建农业果实称重云端系统,成为推动农业数字化转型的必然趋势。云端系统能够实现果实称重数据的实时采集、传输和存储,打破了时间和空间的限制,使得农民和农业管理者可以随时随地获取和管理称重数据\cite{李道亮2012物联网与智慧农业}。通过对海量称重数据的分析挖掘,云端系统能够为农业生产提供更加科学、精准的决策支持,如产量预测、市场趋势分析等,有助于农业生产者更好地应对市场变化,降低生产风险\cite{韩佳伟2022装备与信息协同促进现代智慧农业发展研究}。云端系统还可以与其他农业信息化系统进行集成\cite{刘洋2013基于物联网与云计算服务的农业温室智能化平台研究与应用术},实现农业生产全过程的信息化管理,进一步提升农业生产的智能化水平,推动智慧农业的发展。

基于以上背景,本文研究将深入调研农村电子秤称重的流程,了解农场电子秤称重数据的特点和应用场景,设计并利用软件技术,实现对农场电子秤称重数据的实时采集,并通过云端数据库存储和分析,实现对数据的实时共享和高效分析,为农业生产提供更加科学、精准的决策支持。同时系统还将提供与称重数据相关的果实、员工、采摘作业等信息的管理和可视化展示,帮助农业生产者更好地管理农业生产环境,提高农业生产效率。

\section{国内外研究现状}

在国外,智慧农业的发展起步较早,农业果实称重技术与云端系统的应用研究也取得了较为显著的成果。美国作为农业科技强国,在农业领域广泛应用了先进的传感器技术和物联网技术\cite{赵春江2021智慧农业的发展现状与未来展望}。例如,一些大型农场采用了高精度的电子称重传感器,能够实时采集果实的重量数据,并通过无线传输技术将数据发送至云端服务器\cite{陈学庚2020农业机械与信息技术融合发展现状与方向}。在加利福尼亚州的部分果园,利用智能称重设备与云端系统相结合,实现了对水果采摘过程的精准监控和管理\cite{Ampatzidis2011},农场主可以通过手机或电脑随时查看每个采摘区域的果实产量、重量分布等信息,从而及时调整采摘计划和资源分配。美国还在研发基于图像识别和深度学习的果实称重与分级系统,通过对果实图像的分析\cite{Anisha2019FruitRU},不仅能够准确获取果实的重量,还能对果实的品质进行评估,进一步提高了果实称重与分级的智能化水平。

欧洲在农业果实称重云端系统的研究和应用方面也处于领先地位。德国的一些农业企业开发了集成化的农业生产管理平台,其中包含了果实称重与数据分析模块\cite{Yin2020}。这些平台通过与各种农业设备的连接,实现了果实称重数据的自动采集和实时上传,同时利用大数据分析技术对数据进行深度挖掘,为农业生产提供了全面的决策支持,如预测产量、优化种植方案等\cite{Phate2021}。荷兰则专注于温室农业中的果实称重技术研究,通过在温室中安装传感器网络,实现了对温室作物果实生长过程的全程监测和重量测量\cite{Graaf2004},利用云端系统对数据进行分析和管理,为温室作物的精准种植和高效管理提供了有力保障。

在国内,随着智慧农业的快速发展,农业果实称重云端系统的研究和应用也逐渐受到重视。近年来,国内高校和科研机构在该领域开展了大量的研究工作,并取得了一系列的成果。一些研究团队开发了基于物联网的果实称重系统,通过将电子秤与物联网模块相结合,实现了果实称重数据的无线传输和远程监控\cite{Zhu2013}。在山东的一些果园,应用了这种物联网称重系统,果农可以通过手机 APP 实时查看果实的称重数据,方便了果实采摘和销售管理\cite{Gao2023}。国内企业也积极投入到农业果实称重云端系统的研发和推广中。一些科技公司推出了针对不同农业场景的果实称重解决方案\cite{Ningbo2019},不仅提供了硬件设备,还开发了相应的软件平台,实现了数据的云端存储、分析和可视化展示,帮助农业生产者更好地管理果实称重数据,提高农业生产效率。

然而,目前国内外的研究仍存在一些不足之处。一方面,部分果实称重系统的精度和稳定性有待提高,在复杂的农业生产环境中,容易受到温度、湿度、电磁干扰等因素的影响,导致称重数据出现偏差\cite{汤建华2018}。另一方面,现有的云端系统在数据安全和隐私保护方面还存在一定的风险,随着农业数据的价值不断提升,如何保障数据的安全传输和存储,防止数据泄露和篡改,成为亟待解决的问题。此外,不同地区和不同作物的果实称重需求存在差异,现有的系统在通用性和适应性方面还需要进一步优化,以满足多样化的农业生产需求,比如云端系统需要支持更多物联网通信协议,如 MQTT、HTTP 等,以满足更多不同类型电子秤的通信需求。

\section{研究目标与内容}

本研究旨在设计并实现一个高效、稳定、功能强大的农业果实称重云端系统,以满足现代农业生产对果实称重数据精准管理和深度分析的需求。具体研究目标如下:

一、系统设计与实现:运用先进的信息技术,构建一个基于云端架构的农业果实称重系统,实现果实称重数据的实时采集、传输、存储和处理。确保系统具备良好的用户界面,操作简便,易于推广使用,为农业生产者和管理者提供便捷的服务。

二、多协议兼容与设备管理:使系统支持多种常见的电子秤通信协议,如 HTTP/HTTPS、MQTT 等,能够与市场上不同类型的电子秤设备进行无缝对接,实现数据的准确接收和解析。同时,提供完善的电子秤设备管理功能,包括设备注册、配置、状态监测等,方便用户对电子秤进行统一管理。

三、数据统计与导出:开发强大的数据统计和导出功能,支持从批次、采摘人员、时间等多个维度对称重数据进行统计,生成报表和折线图,便于管理者更直观地分析数据,了解农村情况,从而更好地优化生产策略,提高经济效益。

四、系统性能优化与测试:对系统的性能进行全面优化,确保系统在高并发、大数据量的情况下能够稳定运行,具备良好的响应速度和扩展性。通过一系列的性能测试,评估系统的可用性、每秒查询率(QPS)、并发用户数和响应时间等指标,验证系统是否满足设计要求。

围绕上述研究目标,本研究的主要内容包括以下几个方面:

一、系统架构设计:深入研究云计算、物联网、大数据等相关技术,结合农业果实称重的业务需求,设计出合理的系统架构。包括云端软件前端界面设计、电子秤终端模拟界面设计、后端服务架构、数据存储方案以及通信协议的选择等,确保系统的稳定性、可扩展性和安全性。

二、功能模块开发:根据系统设计方案,开发各个功能模块,主要包括工作服务模块、称重服务模块、用户管理模块、果实管理模块等。在开发过程中,遵循软件工程的原则,采用先进的开发框架和技术,确保代码的质量和可维护性。

三、性能测试与优化:制定详细的性能测试计划,运用专业的测试工具对系统进行性能测试,分析测试结果,找出系统存在的性能瓶颈。针对性能瓶颈,采取相应的优化措施,如优化数据库查询语句、调整服务器配置、采用缓存技术等,不断提升系统的性能。

\section{研究方法与技术路线}

本研究综合运用多种研究方法,以确保研究的科学性、全面性和实用性,包括具体文献研究法、案例分析法、系统开发方法、测试验证法。各研究方法具体内容如下:

一、文献研究法:广泛收集和查阅国内外关于智慧农业、果实称重技术、云计算、物联网等领域的相关文献资料,包括学术期刊论文、学位论文、研究报告、专利文献以及行业标准等。对这些文献进行系统梳理和分析,了解该领域的研究现状、发展趋势以及存在的问题,为本研究提供理论基础和研究思路,避免重复研究,确保研究的创新性和前沿性。通过对相关文献的研究,深入了解现有的果实称重技术和云端系统架构,为系统的设计与实现提供参考。

二、案例分析法:选取国内外具有代表性的农业果实称重项目和智慧农业应用案例进行深入分析,研究其系统架构、功能实现、应用效果以及面临的挑战等。通过案例分析,总结成功经验和不足之处,为本研究的农业果实称重云端系统的设计与优化提供实践参考。例如,对美国某大型农场采用的智能称重设备与云端系统相结合的案例\cite{Anisha2019FruitRU}进行分析,了解其在提高果实称重效率和管理水平方面的具体做法和成效,从中汲取有益的经验。

三、系统开发方法:依据软件工程的原理和方法,本项目将采用瀑布模型的开发方法来组织开发工作。瀑布模型又称生存周期模型,由 B.M.Boehm 提出,是软件工程的基础模型。其核心思想是按工序开发软件,将功能的分析、设计与实现分开,便于分工协作\cite{叶俊民2006软件工程}。农业果实称重云端系统的开发将采用结构化的分析与设计方法,把逻辑实现与物理实现分开,各项软件工程活动包括:制定开发计划、进行需求分析和说明、软件设计、程序编码、测试及运行维护。

四、测试验证法:运用专业的测试工具和方法,对开发完成的农业果实称重云端系统进行全面的测试。功能测试主要验证系统是否满足各项功能需求,如电子秤数据采集、数据传输、数据分析、报表生成等功能是否正常;性能测试主要评估系统在高并发、大数据量情况下的性能表现,包括系统的响应时间、吞吐量、并发用户数等指标;安全测试主要检测系统的安全性,如数据加密、用户认证、权限管理等方面是否存在漏洞。通过测试,及时发现系统存在的问题和缺陷,并进行优化和改进,确保系统能够稳定、可靠地运行。

本研究的技术路线主要包括需求分析、系统设计、功能模块开发、系统集成与测试、系统部署与应用以及研究总结与展望。下面对这些步骤进行具体阐述。

首先是需求分析,调研农业的实际生产流程,了解在果实称重管理方面的实际需求和业务流程,分析现有果实称重方式存在的问题和痛点,明确系统的功能需求、性能需求、安全需求以及用户体验需求等,整理相关的需求文档,为后续的系统设计提供依据。

接下来是系统设计。基于需求分析的结果,结合云计算、物联网、大数据等技术,进行农业果实称重云端系统的架构设计,确定系统的前端界面设计、后端服务设计、数据库设计、数据库存储方案以及通信协议等。根据需求文档所描述的业务流程,设计并书写接口文档,方便前后端的并行开发。对于具体的业务细节,可以利用 UML 图来描述业务的具体流程,比如可以使用时序图和活动图来描述称重的流程,以及使用类图来描述各个模块的关系。在前端设计中,注重用户界面的友好性和易用性,采用响应式设计,确保系统能够在不同设备上正常运行;在后端设计中,注重服务的高可用,提供响应速度良好的接口,提高系统的可扩展性和维护性;在数据库设计中,可以遵循改进后的新奥尔良方法(New Orleans Method),将数据库设计分为需求分析、概念结构设计、逻辑结构设计、物理结构设计、数据库实施和数据库运行和维护这六个阶段,设计出良好、可维护的数据库\cite{苗雪兰2001数据库系统原理及应用教程};在数据存储方面,选择合适的数据库管理系统,如关系型数据库和非关系型数据库相结合,以满足不同类型数据的存储需求;在通信协议方面,选用 HTTP、MQTT 等常见协议,确保电子秤与云端系统之间的数据传输稳定可靠。

完成系统设计之后,即可开始进行功能模块开发。根据系统设计方案,开发各个功能模块。利用前端开发技术,如 Vue、Vite、TypeScript 等,实现系统的前端界面,包括电子秤配置界面、数据查看界面、报表生成界面等;利用后端开发技术,如 Java、Spring Boot、Spring Security 等,实现系统的后端服务,包括工作服务模块、称重服务模块、用户管理模块、果实模块等;利用数据库开发技术,如 MySQL、Redis 等,实现数据的存储和管理;利用通信技术,如 MQTT 客户端库、HTTP 客户端库等,实现电子秤与云端系统之间的数据传输。

完成各个模块的开发之后,即可开始进行系统集成与测试。将开发完成的各个功能模块进行集成,搭建完整的农业果实称重云端系统。制定详细的测试计划,对系统进行全面的测试,包括单元测试、集成测试、系统测试和验收测试等。在测试过程中,记录测试结果,分析测试数据,及时发现并解决系统中存在的问题和缺陷。对系统的性能进行优化,如优化数据库查询语句、调整服务器配置、采用缓存技术等,提高系统的响应速度和吞吐量,确保系统能够满足实际应用的需求。

测试完成,确保没有问题之后,即可进行系统部署与应用。将测试通过的系统部署到生产环境中,进行实际应用。在应用过程中,收集用户的反馈意见,对系统进行持续优化和改进。对系统的应用效果进行评估,分析系统在提高果实称重效率、数据管理水平以及为农业生产决策提供支持等方面的作用,总结经验教训,为进一步完善系统和推广应用提供参考。

最后,进行研究总结与展望:对整个研究过程和结果进行总结,归纳研究成果和创新点,分析研究过程中存在的问题和不足之处,提出未来的研究方向和展望。为农业果实称重云端系统的进一步发展和应用提供理论支持和实践指导,推动智慧农业的发展。

通过以上研究方法和技术路线,本研究旨在构建一个高效、稳定、功能强大的农业果实称重云端系统,为现代农业生产提供有力的支持和保障。

\section{本章小结}

本章主要介绍了课题研究的背景与意义、国内外的研究现状。农业果实称重云端软件是优化农业生产流程的一个重要角色,有利于提高农业生产的效率。本课题就我国目前农业果实称重的发展状况与信息化发展水平,研究并实现了农业果实称重云端软件,支持多协议接入和多维度的数据统计,和农场信息管理无缝对接,带动农业信息化的发展,进一步促进基础体系的形成。

%%%%%%%%%%%%%%%%%%%%%%%%%%%%%%%%

\chapter{系统分析}

\section{系统需求分析}
\subsection{系统功能需求分析}
\subsection{非系统功能需求分析}

\section{系统可行性分析}
\subsection{操作可行性分析}
\subsection{技术可行性分析}
\subsection{经济可行性分析}

\section{系统模块分析}
\subsection{果实称重环节的关键信息}
\subsection{功能模块具体分析}

\section{本章小结}

%%%%%%%%%%%%%%%%%%%%%%%%%%%%%%%%

\chapter{系统设计}

\section{系统设计原则}

\section{系统架构设计}

\section{MVC 与 MVVM 框架模式}
在前后端分离的架构模式下,MVC(Model-View-Controller)设计模式依然存在,但它的角色和应用场景发生了一些变化。前后端分离的架构通常意味着前端和后端的职责和运行环境是分开的,前端通过 API(通常是 RESTful 或 GraphQL)与后端进行通信。

1. 后端中的 MVC

后端依然可以使用 MVC 模式来处理请求和响应。具体来说,后端的 MVC 主要处理以下内容:
	•	Model:表示业务逻辑层或数据模型,通常是数据库实体或服务层中的数据对象。它负责处理数据存取、验证和计算等操作。
	•	View:在传统的 MVC 中,View 是用户界面部分,负责显示数据。而在前后端分离架构中,后端的 View 主要是指服务器端生成的 JSON 或 XML 数据,它不是直接渲染页面,而是以 API 的形式将数据返回给前端。
	•	Controller:处理客户端请求、调用相应的业务逻辑(Model),并最终返回一个响应。Controller 将从 Model 中获取数据,通常会将其转化为 JSON 格式并返回给前端。

在这种情况下,后端的 MVC 模式 用于接收和处理 API 请求,完成业务逻辑并返回数据,而不直接负责页面渲染。后端的 Controller 负责接收前端发送的 HTTP 请求,调用 Model 层进行数据处理,然后通过 API 将数据返回给前端。

2. 前端中的 MVC(或 MVVM)

前后端分离架构中的前端通常使用 MVVM(Model-View-ViewModel)或类似的设计模式,而不完全遵循传统的 MVC 模式。前端通常不负责数据的存储与计算,而是通过 API 从后端获取数据并展示。
	•	Model:在前端中,Model 主要指代数据模型,通常通过 API 获取的数据。例如,前端通过 HTTP 请求从后端获取的 JSON 数据。
	•	View:表示用户界面(UI),前端的页面、组件或视图,由前端框架(如 React、Vue、Angular)渲染。
	•	ViewModel:前端的 ViewModel 是将 Model 和 View 连接起来的部分,负责将数据处理成适合显示的格式,并监听用户的交互操作。前端框架(如 React、Vue)通常在这个层次进行数据绑定与更新。

在前后端分离的架构中,前端不直接处理数据存储或计算,而是依赖后端提供 API 接口来获取数据。前端通过调用后端 API,获取数据并根据这些数据更新 View(页面)。

3. 总结:前后端分离中的 MVC
	•	后端 MVC:后端依然可以使用 MVC 模式,控制数据流、业务逻辑处理和响应生成,但这里的 “View” 不再是传统的页面渲染,而是返回给前端的 API 数据(如 JSON)。
	•	前端 MVVM:前端更倾向于采用 MVVM 或组件化模式,关注如何将后端返回的数据通过 UI 展示出来,且通过 ViewModel 将数据与 UI 绑定。

4. 前后端分离下 MVC 与其他模式的对比

设计模式	适用场景	主要特点	示例
传统 MVC	单体应用或后端渲染页面	适用于前端与后端紧密耦合的应用	JSP、Spring MVC(服务器端渲染)
前后端分离的 MVC	后端通过 API 提供数据,前端渲染页面	后端处理业务逻辑,前端通过 API 获取数据	Spring Boot (后端 API),React/Vue (前端)
MVVM	前端渲染与动态数据绑定	数据绑定和 UI 更新更加解耦	Vue.js、Angular、React (通过 Redux 或 Context)



⸻

5. 实践中的例子
	•	在 Spring Boot 后端项目中,通常会采用 MVC 模式来设计 RESTful API。后端 Controller 接收前端的 HTTP 请求,调用 Service 层(Model)处理数据,返回一个包含数据的 JSON 响应。
	•	在 React 或 Vue.js 前端项目中,使用 MVVM 模式,Model 是从后端 API 获取的 JSON 数据,View 是用户界面,ViewModel 是控制数据展示和用户交互的逻辑代码。

6. 结论
	•	MVC 设计模式在前后端分离的架构中仍然存在,但它的实现方式有所不同。后端通常依然遵循 MVC 模式,而前端更多采用 MVVM 模式或类似的模式来处理用户界面。
	•	后端 通过 MVC 处理业务逻辑、数据和响应,前端 则负责数据的展示和用户交互,数据交互通过 API 完成。

前后端分离架构下的 MVC 设计模式更加注重后端与前端的解耦,前端通过 API 获取数据并进行渲染,从而更好地支持灵活和高效的开发。

\section{主要功能模块的设计}

\begin{figure}[htb]
    \centering
    \includegraphics[width=0.8\linewidth]{../design/out/称重流程时序图.png}
    \caption{称重流程时序图}
    \label{fig:称重流程时序图}
\end{figure}

\section{通信协议选型}

本系统支持 HTTP 和 MQTT 两种通信协议,电子秤可以通过调用 HTTP 接口或者发送 MQTT 消息来完成称重信息的上传。

\subsection{HTTP协议}

HTTP(HyperText Transfer Protocol)即超文本传输协议,是一种用于分布式、协作式和超媒体信息系统的应用层协议。在电子秤系统中,通过调用 HTTP 接口上传称重信息具有以下特点:

一、稳定性和通用性:HTTP 是互联网上广泛使用的协议,具有很高的稳定性和通用性。几乎所有的网络设备和软件都支持 HTTP 协议,这使得电子秤可以很容易地与各种不同的系统进行集成。无论是传统的服务器架构,还是现代的云服务平台,都可以通过 HTTP 接口接收电子秤上传的称重信息。例如,许多企业的内部管理系统、电商平台的库存管理系统等都可以通过 HTTP 接口与电子秤进行数据交互,实现实时的库存更新和物流跟踪\cite{Zhao2016}。

二、易于开发和调试:HTTP 协议的开发和调试工具非常丰富。开发人员可以使用常见的编程语言和开发框架,如 Java、Python、Node.js 等,轻松地实现 HTTP 接口的调用和数据处理。同时,各种网络调试工具,如 Postman、Fiddler 等,可以帮助开发人员快速地测试和调试电子秤与服务器之间的通信,确保数据的准确传输。

三、支持多种数据格式:HTTP 协议支持多种数据格式的传输,如 JSON、XML 等。这使得电子秤可以根据不同的应用需求,选择合适的数据格式上传称重信息。例如,对于需要与第三方系统进行数据集成的场景,可以选择使用 JSON 格式,因为 JSON 格式具有简洁、易读、易于解析等优点,被广泛应用于 Web 开发和数据交换领域。

\subsection{MQTT协议}

MQTT(Message Queuing Telemetry Transport)即消息队列遥测传输协议,是一种轻量级的发布/订阅模式的消息传输协议。在电子秤系统中,发送 MQTT 消息上传称重信息具有以下优势:

一、低带宽和低功耗:MQTT 协议非常适合在低带宽和低功耗的环境下使用。对于电子秤等物联网设备来说,通常需要在有限的网络资源和电池电量下进行数据传输。MQTT 协议通过采用轻量级的消息格式和高效的发布/订阅模式,可以大大降低网络带宽的占用和设备的功耗。例如,在一些偏远地区或者移动环境下,网络带宽可能非常有限,此时使用 MQTT 协议可以确保电子秤能够及时上传称重信息,同时不会对网络造成过大的负担\cite{Jia2015}。

二、实时性和可靠性:MQTT 协议支持 QoS(Quality of Service)级别,可以根据不同的应用需求,选择不同的消息传输质量级别。在 MQTT 协议中,QoS 有三种级别\cite{Jia2015},分别是:
\begin{itemize}
    \item QoS0 – “最多一次”:消息最多传送一次,不保证消息到达目标。适用于实时性要求非常高,但对消息丢失容忍的场景。
    \item QoS1 – “至少一次”:保证消息至少传送一次,但可能会重复。适用于需要消息可靠到达,但不介意重复接收的情况。
    \item QoS2 – “只有一次”:确保每条消息只会传送一次,且无重复。适用于对消息的唯一性和可靠性有较高要求的场景。
\end{itemize}

对于需要实时上传称重信息的场景,可以选择 QoS1 或 QoS2 级别,确保消息的可靠传输。同时,MQTT 协议的发布/订阅模式可以实现实时的数据推送,当电子秤上传称重信息后,订阅了该主题的客户端可以立即收到消息,实现实时的数据更新。

三、易于扩展和集成:MQTT 协议具有良好的扩展性和集成性。可以很容易地与其他物联网协议和技术进行集成,如 CoAP、HTTP、WebSocket 等。同时,MQTT 协议的发布/订阅模式可以支持大规模的设备连接和数据传输,适用于构建物联网应用平台。例如,在智慧农业大棚测控系统中,采用 MQTT 协议将智慧大棚测控系统和阿里云物联网平台结合在一起,通过手机 APP 或 PC 软件访问阿里云服务器数据库,实现了移动终端对农业大棚实时监测和控制\cite{Liang2020}。

\section{数据存储技术}

MySQL + Redis

\section{后台Spring技术栈}

\section{前台Vue技术栈}

\section{服务接口设计}

\section{数据库设计}

\section{本章小结}

%%%%%%%%%%%%%%%%%%%%%%%%%%%%%%%%

\chapter{系统功能实现}
\section{登录界面}
\section{个人信息模块}
\section{工作服务模块}
\section{称重服务模块}
\section{用户管理模块}
\section{果实管理模块}
\section{终端模拟器}
\section{本章小结}

%%%%%%%%%%%%%%%%%%%%%%%%%%%%%%%%

\chapter{系统测试}
\section{系统测试的目的与方法}
\section{测试模型}
\section{测试环境}
\section{主要测试用例}
\section{系统测试结果}
\section{本章小结}

\chapter{总结与展望}
\section{总结}
\section{不足与展望}
\section{创新点}

\end{document}
