\chapter{总结与展望}

\section{论文工作总结}

本文研究深入调研农业生产活动,应用软件工程相关知识理论,归纳与分析具体需求,利用先进软件技术,设计出合理的系统架构并可开发农业果实称重云端软件,实现高性能且稳定可用的智能化农业数据处理流程、高效且丰富的信息管理功能以及多个维度下数据的可视化。论文的工作如下:

1、背景调研和需求分析。调研农业生产活动,分析果实收获和称重入库的具体流程,进行需求分析与可行性分析,其中需求分析包括功能需求分析和非功能需求分析,可行性分析包括技术可行性、经济可行性和社会可行性三个方面的分析。

2、对软件涉及到的技术进行概述。本文列举并简要描述了软件设计和开发过程中涉及到的理论和相关技术,包括 Spring 技术栈、YOLO 目标检测算法、EMQX 网关框架等,对各个技术的原理和优势特点进行了论述。

3、软件系统设计。在需求分析的基础上,选用当下可靠的框架和技术,进行了软件架构设计、数据库设计、称重数据处理流程设计以及农场数据管理功能设计。

4、实现软件各项功能组件。根据设计,对业务功能进行编码实现,完成各个软件模块的联调与集成。

5、软件部署与功能测试。通过使用容器技术组织各个应用,在本地完成部署,然后完成功能测试和非功能测试,对测试结果进行理论分析,给出相关结论。

\section{后续工作展望}

通过软件的设计、实现和测试,本文的农业果实称重云端软件可以满足中小型农场的使用,能够保证数据的准确性和可靠性,为农业生产提供有力的支持。但是,本文的研究仍然存在一些不足之处,主要体现在以下几个方面:

1、软件所实现的 EMQX 静态集群在使用上存在一定的局限性。静态集群在伸缩性和灵活性上较差,无法更高效地增删 EMQX 节点,以满足扩展变更需求。后续工作中,可以考虑基于 Mnesia 之类的分布式数据库,实现一个 EMQX 动态集群,为农业果实称重云端软件提供更加强大的数据处理能力。

2、在数据存储方面存在一定的局限性。随着业务不断扩展,系统产生的数据会以数量级增长,存储方面,本软件采用了 MySQL,但随着数据量的增加,单机数据库的性能瓶颈逐渐显现。为应对这一挑战,后续工作中可以考虑进行数据库分片,通过水平分片将数据分散到多个数据库实例上,以提高查询性能和扩展性。此外也可以考虑进行数据归档与清理,定期归档历史数据,清理过期或不常用的数据,减轻数据库负担。通过这些优化措施,能够有效支持系统的可扩展性和高性能需求。

3、图像识别模型的推理准确度有待提高。本软件使用了来自22种果实的近两千张照片进行模型训练,训练出来的模型性能较好,但是准确度局中。后续工作中,可以考虑增大数据集或优化训练流程,以训练出更高推理准确度的模型。
